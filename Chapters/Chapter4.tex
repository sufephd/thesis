% Chapter 4

\chapter{Matching}  % Main chapter title

\label{Chapter4} % For referencing the chapter elsewhere, use \ref{Chapter4} 

%----------------------------------------------------------------------------------------

% Define some commands to keep the formatting separated from the content 
%\newcommand{\keyword}[1]{\textbf{#1}}
%\newcommand{\tabhead}[1]{\textbf{#1}}
%\newcommand{\code}[1]{\texttt{#1}}
%\newcommand{\file}[1]{\texttt{\bfseries#1}}
%\newcommand{\option}[1]{\texttt{\itshape#1}}


\section{Introduction of the basic concepts}



%I would like to introduce Coq Proof Assistant for this purpose, 
% for there are many theorems in matching theory which can be proved
% more strictly and without errors with its help. My motivation comes
% from the fact that proof error is easy to creep into human proofs 

% and most proofs can only be best understood and checked by the original author when the body of theorems and propositions in matching theory become more and more abstract in quality and large in
% quantity. 
 
% The adoption of Coq for this subject is an idea inspired by the 2002 Fields Medal Winner Vladimir Voevodsky's recent project Univalent foundations for mathematics. He uses Coq to formalize the proofs of theorems
% in homotopy theory and try to build another foundation of mathematics which can be viewed as equal to the set theory but is much easier for the computer to check the correctness of proofs. 
 
% first, we consider the easy one. Man woman marriage market as defined in the classical work of (Alvin Roth and SotoMayor). 
% We define these markets as a type with m men and n woman as follows
% Inductive Market (m n: nat) :=
% |mkt: Men n -> Women m -> Market m n
We consider two-sided matching in this chapter. There was a monograph dedicated to it, see \parencite{Roth1990}. However, we will focus on application of the theory to Chinese college admission mechanism, and use some notions that have recently developed. In this section, some of the most important theoretical concepts and results are introduced. 

In two sided matching, the two
distinguishable sides of matching are assumed to be men and women,
students and schools, workers and employers, tenants and houses,
etc. Here, students and colleges are chosen as the economic
background. For this specific topic, there
is \parencite{Sonmez2003}  which give a description of the literatures
in this field till that time. 

Now A description of the model and relevant notations that will be used in this chapter. There is a finite set of students and a finite set of colleges denoted by  $S = \{1,...,n\}$ and $C = \{1,...,m\}$, respectively. For any $j \in C$, college $j$ has $q_j$ quota, which is college $j$'s seats for students. Students wish to enter at most one college and have an option not to enter any college at all. This outside option is formally represented by a null college, denoted by $0$. This null college has unlimited quota, ie, $q_0 = + \infty$. 

A matching(sometimes also called assignment or allocation) is a
mapping $\mu : S \rightarrow C\cup\{0\}$ such that for any $j \in C$,
$|\mu^{-1}(j)| \leq q_j$. Denote by $\mu(0)$ the set of colleges with
seats that is not assigned to any student at matching $\mu$. The null
college is always included in this set, as its seat supply is
unlimited. Hence:
\[ \mu(0) = \{j \in C : |\mu^{-1}(j)| < q_j\}\cup \{0\}\]


We assume students have preferences on colleges.  Denote by $R_i$ student $i$'s preference on the set of colleges $C \cup \{0\}$. The corresponding strict preference and indifference relations are denoted by $P_i$ and $I_i$, respectively. The meaning of the notation is as follows: if $c,c'\in C \cup \{0\}$ and $c R_i c'$, then student $i \in S$ weakly prefers college $c$ to college $c'$. Change ``weakly prefers'' to ``strictly prefer'' we get the meaning for the strict $P_i$. Change ``weakly prefers'' to ``are indifferent'' we get the meaning for the indifferent $I_i$. Preferences are assumed to be rational in the sense that for all $i \in S$, $R_i$ is complete, reflexive and transitive. A preference profile is a list $ R = (R_1,...,R_n)$ of the students' preferences. All the discussions here and after are based on the strict preference assumption. That is , if $c I_i c'$, then $c=c'$.\footnote{Because the students have strict preference, a college can only be indifferent to itself in any student's preference. No two things are exactly the same, and indifference are usually the result of inadequate information, so here we assume aspects of the university are known to the extent that the colleges can be ranked by a student.}

 Here, a priority structure $\pi$ is assumed for the colleges, that is, for each college $j \in C$, there is an exogenously given strict priority-order $\pi_j$. Formally, $\pi_j: S \rightarrow S$ is a bijection where the highest-ranked student  is the student $i\in S$ with $\pi_j(i)=1$, the second highest ranked student $i' \in S$ has $\pi_j(i')=2$, and so on. When $\pi_j$ is the same for all $j\in C$, we get the serial dictator matching mechanism. A priority structure is a list $ \pi = (\pi_1,...,\pi_m)$ of the colleges' priority-order. For the concept of pairwise stable, this priority structure is enough. \footnote{For group stable or core, further definition of college preference on groups of students should be provided. And for substitutable college preferences which is an implied assumption of this chapter, the two concepts coincide. See Proposition 6.4 in \parencite{Roth1990} for more details.}

 A matching $\mu$ is pairwise stable(or called priority respecting as in \parencite{Svensson2014}) if there is no student $i \in S$ who strictly prefers some college $j$ to $\mu(i)$ and $\mu^{-1}(j)$ contains some other student $i' \in S$ who has lower priority for college $j$ than student $i$(i.e., $\pi_j(i')>\pi_j(i)$), and furthermore, all students weakly prefer their assigned college seats to any unassigned seats in $\mu(0)$(seats from the same college are obviously equal). We'll simply use stable to mean pairwise stable later in the chapter.

 Formally ,the definition of stableness is as follows.

\begin{definition*}(stableness)
A matching $\mu$ is stable for a given preference profile $R$ and a given priority structure $\pi$, if :

(i) for all $i,i' \in S$, $\mu(i')P_i \mu(i)$ only if $\pi_{\mu(i')}(i') < \pi_{\mu(i')}(i)$;

(ii)for all $i \in S$, $\mu(i)R_i j$ if $j \in \mu(0)$, where $\mu(0) = \{j \in C : |\mu^{-1}(j)| < q_j\}\cup \{0\}$.

\end{definition*}

Condition (i) is called fairness condition, with the meaning that
there is no justified envy\footnote{justified envy means that you envy somebody
else's college and according to your priority in that school, you
should be admitted prior to that person}. Condition (ii) is the combination of
individual rationality and non-wastefulness. Rationality means that
the current assignment of college for a student must be weakly better than the outside
option of null college 0. Non-wastefulness means that the current
assignment of college for a student  must be weakly better than any
unocuppied college seat. For more detailed discussion of these concepts
, see also \parencite{Sonmez1999}. 

The next important concept is Pareto efficiency for students or simply
efficiency. A matching is pareto efficient if no other matching can
make at least one student get a strictly more preferred college while
no student get a strictly less preferred college. 

Formally, we have the following definition.
\begin{definition}
A matching $\mu$ is efficient for a given preference profile $R$, if
for all $\mu'$ such that there exists $i \in S$ satisfying $\mu'(i)
P_i \mu(i)$, then there must be some other $i' in S$ satisfying
$\mu(i') P_{i'} \mu'(i')$
\end{definition}

It is clear that efficiency for students does not take into account of college priority while stableness does need it. Stableness is priority respecting and includes a flavor of fairness in it. Now that we have some idea about matching, we begin to investigate mechanism that can lead to stable and efficient matching. An important difference between matching mechanism design and other mechanism design such as auction is that efficiency is usually not the only requirement, stableness or priority respecting is at least of equal importance. We will illustrate this later in examples.

Why do we care about priority? In the previous chapter, we only care about efficiency. which means to maximize the sum of all participants' utilities. However, there is a well known saying that the whole is more than the parts put together. Priority should belong to the more-than part of the whole. suppose that only students entering their most preferred college will study hard, and any student would feel the same level of happiness when they enter their most preferred college but those with high scores will contribute more to the society, so high scored students should have priority. However just by the limited maximizing sum efficiency criterion we would send arbitrary student to their most preferred college, while from the society as a whole giving the high scored students high priority is a better choice.\footnote{I never believe this myself, so it is only a possible reason coined to show that priority may matter to the social welfare; other possibility of it may be fairness itself is an component of social happiness.}

When it comes to mechanism design, the usual problem of unilaterally misreporting one's private information(here is one's preference) emerges. In the literature, a mechanism that is immune to such problem is often called strategyproof. Viewing the reporting of one's preference under a mechanism as a game among the students, strategyproofness  means that it is a dominant strategy to report one's true preference. 

There is also a related concept called group-strategyproof. What is the connotation of this concept? It means that no group can get a pareto improvement by unilaterally changing the reports of students in this group under the mechanism. Viewing the reporting of one's preference under a mechanism as a game among the students, group-strategyproofness means that it is a core equilibrium for everyone to report the private true preference.



\section{Popular matching mechanisms}
\label{equivalent direct mechanisms}
There are many matching mechanisms. Three kinds are
most studied. They are the Deferred Acceptance Mechanism(DA), the Top
Trading Cycle Mechanism(TTC), and the serial dictatorship mechanism(SD).
These three mechanism are all related to the later analyzed Chinese
student-college matching mechanism, especial the DA mechanism. Why are they most popular? 
One reason is that all of these mechanisms are
strategyproof. Therefore the reported preference profile are true preferences
for rational agents. 



\subsection{Deferred Acceptance Mechanism}
In the seminal work \parencite{Gale1962}, DA mechanism is first
proposed. This mechanism is the first and probably the most studied
mechanism in modern matching theory. 

In the setting of student-college matching, it is also called
student-optimal stable mechanism (SOSM) for it always finds the stable
matching that is most favorable to each student. Its outcome can be
calculated via the following Deferred
Acceptance (DA) algorithm for a given problem:

 Step 1: Each student applies to his or her favorite school. For each
college $j$, up to $q_j$  applicants who have
the highest college $j$ priority are tentatively assigned to college $j$. The remaining applicants are
rejected.

Step k ($ k \geq 2$ ): Each student rejected from a college at step k -1 applies to her next favorite college.
For each college $j$, up to $q_j$  students who have the highest college
$j$ priority among the new applicants and those tentatively on hold from an earlier step, are
tentatively assigned to college $j$. 
The remaining applicants are rejected.

The algorithm terminates when no student applies to any new college in
some step. That is, every  student is either tentatively placed to a college
or has been rejected by every college that is better than null college
in his or her preference list. 

We list important properties concerning the DA mechanism here.

Under strict priority( that is, no two colleges have  the same utility
level for a student.), the following properties
hold. \parencite{Gale1962} first proposed the two theorems and proved
them. The proofs for the theorems are short and elegant, and therefore
provided in Appendix \ref{Appendix_D}.

\begin{thm*}(Gale-Shapley)
The matching given by DA mechanism is stable.
\end{thm*}

\begin{thm*}(Gale-Shapley)
Every student is at least as well off under the assignment given by
the DA mechanism as he would be under any other stable assignment.
\end{thm*}

These two theorems can be combined into one concise statement. The SOSM
matching produced by the DA mechanism is the optimal stable matching.
An alternative way to express this is that if you want to find a
matching that is stable, the SOSM matching selected by the DA  mechanism is the most
efficient matching(constrained efficiency).

\begin{thm*}
\label{thm1}
The DA mechanism is strategy-proof for the students.
\end{thm*}

The proof of it is complex and need more notions, please see \parencite{Roth1982a} or \parencite{Roth1990}for proof.
However, the DA mechanism is not pareto efficient.  This fact can be shown by an example.

\begin{example}
There are two colleges $A$ and $B$, each with 1 seat for students. A null college $0$ with unlimited seats and accepts who ever applies for it. There are
3 students 1,2,3. The preferences of the students are
as follows.

    \begin{center}
      \begin{tabular}{|c|c|c|}
        \hline
        $1$ & $2$ & $3$ \\
        \hline
        $A$ & $B$ & $A$ \\
        
        $B$ & $A$ & $B$ \\

        $0$ & $0$ & $0$ \\

        \hline
        
      \end{tabular}
    \end{center}

The two colleges' base priority structure is the same as in the following
table

 \begin{center}
      \begin{tabular}{|c|c|}
        \hline
         $A$ & $B$ \\
        \hline
        $2$ & $1$\\
        
        $3$ & $3$ \\

        $1$ & $2$ \\
       
        \hline
        
      \end{tabular}
    \end{center}
 
The DA mechanism is run round by round as in the table below.

 \begin{center}
      \begin{tabular}{|c|c|c|c|}
        \hline
        &$1$ & $2$ & $3$\\
        \hline
       round1&   & $B$ & $A$ \\
        
        round2&$B$ &  & $A$ \\

        round3&$B$ & $A$  & \\

        round4&$B$ & $A$ & $0$\\
        \hline
        
      \end{tabular}
    \end{center}

Because 3 is the only student not tentatively accepted after 3 rounds,
and 3 has been rejected by both A and B, the DA algorithm with quota
terminates with student 1 entering B, 2 entering A, and 3 entering null college $0$.

Now consider the following alternative preference profile.
\begin{center}
  \begin{tabular}{|c|c|c|}
    \hline
    $1$ & $2$ & $3$ \\
    \hline
    $A$ & $B$ & $0$ \\
    
    $B$ & $A$ & $A$ \\

    $0$ & $0$ & $B$ \\

    \hline
    
  \end{tabular}
\end{center}

With this profile, the DA mechanism is run as below.

\begin{center}
  \begin{tabular}{|c|c|c|c|}
    \hline
    &$1$ & $2$ & $3$\\
    \hline
    round1& $A$  & $B$ & $0$ \\
    
    \hline
    
  \end{tabular}
\end{center}

Just 1 round, and the DA process is terminated.





\end{example} 

\begin{remark}
 If a mechanism is able to produce every possible matching results, then not pareto efficient obviously means that the mechanism is not group-strategyproof, since reporting the profile that can result in a pareto improvement assignment is a profitable group-deviation.
\end{remark}


%\subsection{Top Trading Cycle Mechanism}
%This mechanism is given in \parencite{shapley1974}. It is also an
%important mechanism and being studied and modified by scholars across
%the matching area since 1974 

%In the student-college matching setting, the TTC matching result is
%determined after the following process.


%\subsection{Serial Dictatorship Mechanism}

\section{Analysis of Chinese college admission mechanisms}

As an application of the matching theory, we will focus
on the Chinese college admission mechanisms in this section.

The Chinese college admission is a centralized matching mechanism with
an entrance exam taken national wide as a means of student-priority
deciding tool for the colleges.
Students from high school in China need to take an entrance exam to
decide the student priority for colleges, and the priority structure
is also influenced by the application form that the students write to
inform authority their own preferences. Starting from the early 1950s,
the mechanisms adopted has gone
through series of reforms,  and what kind of mechanism should be adopted is
currently still in debate.

The college entrance exam, or Gaokao in Chinese Pinyin,  forms the 
foundation of the priority determination process for the current
admission system.  There are millions of high school seniors compete
for  seats at various colleges and universities in China each
year. The matching of students to these colleges and universities has
profound implications for the education and labor market outcomes for
these students.  The fairness and efficiency of this matching process,
is of great importance to the society.

Many Chinese scholars have writen articles discussing problems related
to Gaokao. \parencite{Zhong2004} has done the efficiency comparison of
the three preference reporting  mechanisms, i.e. reporting preference
before the exam, after the exam but not knowing the exact scores,  and
after the scores are known.  This kind of study apparently focused on the
strategical interaction of studnets in the games, not on how to induce
the truthful reporting.  As to the strategical importance of
preference reporting,  Haifeng Nie has written many articles.  Both \parencite{Nie2007a}
and \parencite{Nie2007b} has stressed the point that a good score may
not be as useful as a good strategy for reporting the preference of
the university. This somewhat strange phenomenon has motivated me
in studying the issue of reducing manipulation and inducing better
and fairer results.

The Chinese  matching mechanisms has recently caught attention of overseas scholars from
the economic field of matching. \parencite{YanChenJPE} has
done much to analyze theoretically the sutleties of the different
mechanisms adopted, and has gone even further to design experiments
for studying features of these mechanisms,
see \parencite{YanChen2016}. 

\subsection{A simple model of the Chinese student-college matching}

The mechanisms are hard to analyze with all the real procedures on,
so we will deal with the equivalent direct mechanisms of the mechanisms actually in use.

% Another reason
According to Revelation  Principle,  if a social result can be
implemented in dominant strategy equilibrium in a mechanism, then it
can be truthfully implemented in a direct revelation mechanism. For ex post equilibrium,
it is also true as proved in Appendix \ref{Appendix_A}. Actually, Revelation Principle
has not told all that has been shown in the proof. In the proof, we establish a direct revelation mechanism that
for each strategy profile in the original mechanism, there is strategy profile in the direct revelation mechanism that has the same
result: $g(s_1,\cdots,s_n)=h(\sigma^1(s_1),\cdots,\sigma^n(s_n))\in F(s_1,\cdots,s_n)$. Then we showed that the ex post equilibrium results in the
original mechanism is also an ex post equilibrium results in this direct revelation result. Now we will give this derived direct revelation mechanism a name, it will hereafter be called equivalent direct mechanism of the original mechanism. Remember that we have described an assets_inheritance problem in the examples section \ref{assets_inheritance} in the Introduction Chapter. The second mechanism is the equivalent direct mechanism of the first mechanism. 
%For designing new
%mechanisms, it seems that considering only strategyproof direct mechanism is
%enough.

Therefore when studying ex post equilibrium results for existing mechanisms which are not direct revelation mechanism,
we can use a equivalent direct revelation mechanism with the same ex post equilibrium results for narrational convenience\footnote{We put the mechanisms in the same mechanism class makes it easier to do comparison.}  To aid narration and analysis,  a notion of
successful manipulation  which means a best report choice given others' report  is frequently used in later analysis of the strategical interaction of students in the Chinese college admission mechanisms. \footnote{ It is essentially the best response concept in Nash equilibrium but sounds more fluent when used
in description of the strategical interactions among the students.}

Now, let's introduce two most important mechanisms: sequential mechanism, parallel mechanism. The sequential mechanism, or Boston or Immediate Acceptance (IA) mechanism,
had been the only mechanism used in Chinese student assignments both at the high school
and college level \parencite{Nie2007b}. However, this mechanism has a serious limitation: “a good score in
the college entrance exam is worth less than a good strategy in the ranking of colleges” \parencite{Nie2007a}. In our paper's wording, a successful manipulation is often more important than your score level. The problem arise from the special priority structure in the mechanism. Priority of a student in a college is based on your reported preference list firstly and based on your score secondly, a dictionary order that put too much emphasis on preference reporting. Given in \parencite{Nie2007b}, and also cited in \parencite{YanChenJPE}, one parent has the following explanation of the problem.

``My child has been among the best students in his school and school district. He
achieved a score of 632 in the college entrance exam last year. Unfortunately, he was
not accepted by his first choice. After his first choice rejected him, his second and third
choices were already full. My child had no choice but to repeat his senior year.''

To alleviate this problem of high-scoring students not being accepted by any universities, the
parallel mechanism was proposed. 
In the parallel mechanism, students select several
“parallel” colleges within each choice-band. The priority structure is changed. The priority of a student in a college of the same choice-band is the same. A more detailed description from \parencite{YanChenJPE} is cited here for the importance of fully understanding the mechanism.

``a student’s first choice-band may
contain a set of three colleges, A, B, and C while her second choice-band may contain another set
of three colleges, D, E, and F (in decreasing desirability within each band). Once students submit
their choices, colleges process the student applications, using a mechanism where students gain
priority for colleges they have listed in their first band over other students who have listed the same
college in the second band. Assignments for parallel colleges listed in the same band are considered
temporary until all choices of that band have been considered. Thus, this mechanism lies between
the IA mechanism, where every choice is final, and the DA mechanism, where every choice is
temporary until all seats are filled.''

In 2003, Hunan fisrt implemented the parallel mechanism, allowing 3
parallel colleges in the first choice-band(or called group), 5 in the second choice-band, 5 in the third choice-band, 5 in the fourth choice-band, and so on, see \parencite{YanChenJPE}.  The parallel mechanism is soon to be widely perceived as having improved allocation outcomes for students. Citing a parent from a newpaper report \footnote{Li Li. “Ten More Provinces Switch to Parallel College Admissions Mechanism This Year.” Beijing Evening News,
June 8, 2009.},

``My child really wanted to go to Tsinghua University. However,  in order not to
take any risks, we unwillingly listed a less prestigious university as her first choice.
Had Beijing allowed parallel colleges in the first choice band, we could at least give
Tsinghua a try.''

 In the paper  \parencite{YanChenJPE}, they have formulated the IA
 mechanism, parallel mechanism, and DA mechanism as a parametric
 family of application-rejection mechanisms. That is a correct
 description. However, I would like to describe all the three mechanisms
 as generalized DA mechanisms with different priority structures. The IA mechanism
 and parallel mechanism as kinds of DA are very special. The priority
 structure are closely related with the  reported
 preference. The underlying idea for the colleges may be ``I like
 those who like me''. This is where manipulation can creep in.

Formally, under the setting of student-college matching, we have the following definition.

\begin{definition}
A mechanism of matching is non-manipulable, if given any preference
$R$, and priority structure $\pi$, $\mu = h(R, \pi(R))$ and for any $i
\in S$, denote $h(R_i',R_{-i}, \pi(R_i',R_{-i}))$ by $\mu'$, then we have $\mu(i)R_i \mu'(i)$.
\end{definition}

We have the following proposition.

\begin{prop}
IA and parallel mechanisms are manipulable. DA is non-manipulable.
\end{prop}

%The proof is relegated to Appendices.
 
Now let us define the concepts of efficient mechanism and stable
mechanism in the setting of student-college matching.

\begin{definition}
A mechanism is efficient if for any reported preference profile $R$ and
priority structure $\pi$, the resulting matching result is pareto
efficient according to the $R$.
\end{definition}


\begin{definition}
A mechanism is stable if for any reported preference profile $R$ and
priority structure $\pi$, the resulting matching result is stable according to the $R$ and $\pi(R)$.
\end{definition}

According to these definition, we have the following proposition.

\begin{prop}
IA is efficient. IA , parallel and DA are all stable.
\end{prop}

The proof is relegated to Appendices.

\begin{remark}
The efficiency and stableness concept here is apparently not very useful if
the reported $R$ is not true. What we are concerned is the status of the matching
result in the perspective of true $R$.
\end{remark}

Here I would like to view the mechanisms from a new perspective. The concept is from ex post Nash equilibrium. For ex ante Nash equilibrium or Bayesian Nash equilibrium, there is a  best reponse concept.
A best response is a strategy for your type that maximize the ex ante payoff given the others' probability distribution on all probable types and each type's strategy choice. An ex ante Nash equilibrium or
Bayesian equilibrium is a strategy profile in which every player type chooses the best response. 
For ex post equilibrium,  a successful manipulation is similar in meaning to a best response to  ex ante Nash equilibrium. It means that given the other students' actual choice(action), the action you have
chosen is best to your interest. It deprives the intermediary probability distribution structure. This way we can concentrate on the final assignment or allocation result. In real life situations, 
we have to use probability to make decisions ex ante, but when we can avoid it in mechanism design, we had better design a sure way to implement a certain social criteria or goal, not relying on 
probability distribution. The only exception should be entertainment activity like mahjong, poker games, or lottery where no prescribed social goals need to be fulfilled. Here, the ex post should not be taken 
literally, it does not mean that 
evaluating the action after some particular time. Actually, it means that assuming that every action of the other players have been known, the player still insists his choice due to its optimality with respect
to the others' actual action or choice. In fact, it may happen that the player will never know all the actions of the other players and therefore there is not a time point after which all information is revealed. 
The important thing is that the action or manipulation can be called successful when all information is known.
Formally, we give the following definition in the student-college matching setting. 
\begin{definition}
A student i has a successful manipulation 
if let $R$ denote the true preference, $R'$ the reported preference, $\mu' = h(R')$, then for any $R_i''$, let $\mu''= h(R_i'',R_{-i}')$, we have  $\mu'(i) R_i \mu''(i)$
\end{definition}

Here, I would like to explain this seemingly trivial concept. I think it is even more basic than ex post Nash Equilibrium. In fact, I extract it from the definition of ex post Nash equilibrium and find it more applicable than Nash equilibrium in my every day life. Its definition is in fact half of the Nash Equilibrium.

You may say that it is just the best response function. Almost, best response function is a way of getting a successful manipulation when the others' action choice is known. But calling it a successful manipulation is more appropriate in most cases as this concept does not assume that you know the payoff or the strategies of the others. 
Nash equilibrium usually assumes that the payoffs are known. That is, it may not be a purposeful response but just happen to be successful and optimal. It is an ex post equilibrium but it maybe the case that 
nobody knows the payoffs of the game fully. Here, our ``ex post'' requires that the strategies to be considered must be pure.

In this chapter, I only consider deterministic game forms and only consider pure strategies. 

Why this concept deserves a special mention? Well, that is because in most real life situation it is the player who always choose a successful manipulation wins the game. 
\begin{example}
Consider a 
\end{example}

 
However, when you have clear social criteria in mind, it is not good to devise a mechanism in which the social criteria are satisfied only when every agents have a successful manipulation. Not everyone 
is adapted at manipulation.


In this chapter, I find that in the Chinese style college admission, the Boston mechanism, the parallel mechanism and the DA mechanism have the same assignment result in their expost equilibrium. However, it is not 
easy to achieve the expost equilibrium for the Boston mechanism and the parellel mechanism because they are not direct mechanisms while the DA can be seen as an almost direct mechanism which makes the ``truthful
revealation'' the expost equilibrium strategy and easy to find for every student.

Then we can have an important concept defined now.
\begin{definition}
under a preference reporting direct mechanism $h(.)$, and students report some preference profile $R$, therefore the matching result is $\mu=h(R)$.
If all students have successful manipulation in this case, then the matching result $\mu$ is called achievable with perfect manipulation under the mechanism $h(.)$.
\end{definition}

In fact, the concept is just a special Nash implementation in
an indirect mechanism. The difference is that when in a direct
mechanism, you nash implement a result and know whether it satisfies a
social goal $f(p)$, but in a indirect mechanism even if you know that
is successfully manipulated by everyone, you do not know if it
satisfies a social goal $f(p)$, because you do not know the economics
environment $p$!


If you see carefully, you will know that this concept is  a kind of ex post nash
implementation in the particular case of matching mechanism.  For
general priority structures, what
kind of matching results can be achieved with perfect manipulation
under the IA mechanism, see \parencite{Ergin2006}; what kind of
matching results can be achieved with perfect manipulation under the
parallel mechanism, see \parencite{YanChenJPE}.  In this chapter, we
focus our attention on a very important kind of priority structure,
the score-based priority structure and discuss what kind of matching
results can be achieved with perfect manipulation under such
score-based priority. 

DA mechanism just use the score high-low comparison to determine
priority of students for every college. The more score you get, the
high priority you get.

To determine a student $i$'s priority in college $j$, parallel
mechanism fisrt group the colleges according to $i$'s reported
preference list. For example, in Tibetan's parallel mechanism, every
group has 10 colleges, then you group the first 10 together and label
every college in this Group 1, group the second 10 together and label
every college in this Group 2, and so on. A student's priority in a
college is determined in a dictionary order. First, compare the group
number of the college for the student. For instance, if Beijing
University is in the first 10 college group for a tibetan student, 
then this student has a higher priority in Beijing Unversity than
tthose who has not put  Beijing University in the first 10 college
group. If two students put the college in the same Group $n$, then
comes the second procedure of comparison. The student with higher
score(or high original priority) has higher priority. IA mechanism is
just a parallel mechanism with every group size equal to $1$.


We have the following important theorem.

\begin{thm}\label{same}
If every college has the same base priority for the students, and only
modify the priority structure as is necessary for the IA and parallel
mechanism, then the only matching achievable with perfect
manipulation under the IA and parallel mechanism  is the same with the one under  the DA mechanism, which is  unique , efficient and stable. 
\end{thm}

To get this result, we first propose a property of the DA mechanism which is convenient for discussion of its matching result. 

\begin{prop}
 The final
matching result of the DA mechanism is not influenced by the number and order of applications of the students in each step, as long as there is some student in each step  who applies for the college that is his or her favorite among the colleges having not rejected him.
\end{prop}

Because of the finiteness of the problem, any mechanism with the above requirement will end, and we can say that the result is the same as the ordinary DA according to this  proposition. 
%The proof is relegated to the appendix C. 
We call this group of mechanisms as the DA family of mechanisms.

As a direct result of the above propostion and the optimal stableness and non-manipulability, we have the following corollary.

\begin{corollary}
The DA family of mechanisms are all optimally stable, and non-manipulable.
\end{corollary}

\begin{proof}
Now is the proof of the above theorem \ref{same}.
In fact, the Serial dictatorship mechanism is just a member of the DA family of mechanisms and has the property mentioned in the above corollary. Therefore, in the following discussion, we take the Score-based serial dictatorship mechanism for producing the DA matching result.

Now we give a intuitionistic proof of the important theorem. The perfect manipulation of DA mechanism is just non-manipulation, that is, stating the true 
preference according to \ref{thm1}. For the parallel mechanism, we would first prove that for everyone to report as his or her favorite  the college that the DA matching mechanism assigns is a perfect manipulation. 

To prove perfect manipulation, we only need to show that every student has got a successful manipulation by doing so. Using induction on the students' score list(without loss of generality, assuming no same scores).

First, the student with the highest score, anounce his or her true favorite college as his or her favorite, has successfully manipulated. His or her true favorite is obviously the DA matching result. Therefore, this top student has done a successful manipulation by reporting as his or her favorite  the college that the DA matching mechanism assigns. 

Second, for  an arbitrary student $i$. Suppose every student higher in the score list hasreported as his or her favorite  the college that the DA matching mechanism assigns. Then since the DA result is the same as the score-based serial dictatorship result, these students will get what they get in the DA. Now the student $i$ can only expect as the best the DA result, and can indeed get it if reporint it as his or her favorite. Therefore reporting  as his or her favorite  the college that the DA matching mechanism assigns is a successful manipulation for him.

Having proved that the DA matching result is achievable with perfect manipulation, next we prove the uniqueness of the achievable matching result with perfect manipulation. We use terms of the DA matching result and the score-based serial dictatorship matching result interchangeably since they are the same in our setting.
The proof of unique. Take any other matching result ,we would prove that it is not acheivable with perfect manipulation. Because the score-based serial dictatorship matching result is pareto-efficient, and the preference is strict, in any other matching result there must be some students who are worse off thanin the DA matching result. There must be a student with highest score among them, then we would like to show that this student has not successfully manipulated. Still using induction, 

first, if this student is the top scored student, then he or she definitely has not successfully manipulated because if he or she report whatever college as his or her first choice, then he or she will get it.

second, if all the higher scored students has successfully manipulated, then they will get the score-based serial dictatorship matching college, so this student can get the score-based serial dictatorship matching result. If he or she gets a worse college, he or she definitely has not successfully manipulated.
 
Done.
\end{proof}

Now let us go to the next subsection where we put some details on.

\subsection{Student-college matching with affirmative actions of bonus-score
  and quota}

In this subsection, we would like to consider extending the previous
model to cover affirmative actions of bonus-score and quota.
When there are bonus-score given to some student, if the bonus-score
is accepted in every college, then we can just take his or her total score as
original score + bonus-score. The rest is the same as the serial
dictatorship case analyzed in the previous subsection. However, if
there is difference between the accepted bonus-score
among colleges, then the priority structure again goes into the
chaotic state that the score-based serial dictatorship mechanism has
successfully avoided in our previous simple model.

Next is the more discussed affirmative action of quota, 
A prevalent affirmative action policy in school choice limits the number of ad-
mitted majority students to give minority students higher chances to attend their
desired schools. To circumvent the inefficiency caused by majority quo-
tas, \parencite{Hafalir2013} offered a different interpretation of the affirmative action policies based on
minority reserves. With minority reserves, schools give higher priority to minor-
ity students up to the point that the minorities fill the reserves.
we would like to adopt the minority reserve definition
in \parencite{Hafalir2013} with a little extension. we will give every group(including the majority group) a reserve (or
called quota, whatever) that gives a student of this group priority
over other group whenever the reserve number has not been reached.
Using this definition of quota, we give the following enlightening
example.

\begin{example}
There are two colleges $A$ and $B$, each with two seats for students. There are
two students groups $a$ and $b$. Students $1,2,3$ are in group $a$ and
students $4,5$ are in group $b$. The preferences of the students are
as follows.

\begin{center}
  \begin{tabular}{|c|c|c|c|c|}
    \hline
    $1$ & $2$ & $3$ & $4$ & $5$\\
    \hline
    $A$ & $A$ & $A$ & $B$ & $B$ \\
    
    $B$ & $B$ & $B$ & $A$ & $A$ \\
    \hline
    
  \end{tabular}
\end{center}

As can be seen, the students in group $a$ all prefer college $A$ while
the students in group $b$ all prefer college $B$. 

The two colleges' base priority structure is the same as in the following
table

\begin{center}
  \begin{tabular}{|c|c|}
    \hline
    $A$ & $B$ \\
    \hline
    $1$ & $1$\\
    
    $2$ & $2$ \\

    $3$ & $3$ \\

    $4$ & $4$ \\

    $5$ & $5$ \\
    \hline
    
  \end{tabular}
\end{center}
The quota structure
of A and B are both 1 seat for group $a$, 1 seat for group $b$. Now
the DA mechanism is run round by round as in the table below.

\begin{center}
  \begin{tabular}{|c|c|c|c|c|c|}
    \hline
    &$1$ & $2$ & $3$ & $4$ & $5$\\
    \hline
    round1& $A$ & $A$ &  & $B$ & $B$ \\
    
    round2&$A$ & $A$ & $B$ & $B$ &  \\

    round3&$A$ &   & $B$ & $B$ &  $A$\\

    round4&$A$ & $B$ &  & $B$ &  $A$ \\
    \hline
    
  \end{tabular}
\end{center}

Because 3 is the only student not tentatively accepted after 4 rounds,
and 3 has been rejected by both A and B, the DA algorithm with quota
terminates.

Now consider the IA mechanism. If the quota priority is first compared
and considered, then the result is the same as DA. This fact can be
easily verified and we omit it here.  If the reported preference
order is first considered, as the name of Instant Accept is
indicating, then the result is as the table below.

\begin{center}
      \begin{tabular}{|c|c|c|c|c|c|}
        \hline
        &$1$ & $2$ & $3$ & $4$ & $5$\\
        \hline
       round1& $A$ & $A$ &  & $B$ & $B$ \\
        
        \hline
        
      \end{tabular}
    \end{center}

\end{example}

From the above example, we see that the matching result of DA mechanism with quota is not pareto efficient for the students. In fact, the comparison of the tables' 
matching results shows that IA 
mechanism's result is a pareto improvement. Student $2$ and $5$ gets strictly better result in IA mechanism than in DA mechanism. Why the efficiency result of the
the previous subsection lost. The reason is that the efficiency in previous subsection comes from the serial dictatorship nature of DA in that special priority
 structure. Now with quota included, the priority structure is changed and that property is lost in the process. However, the resulted matching is still priority-respecting(as \parencite{Svensson2014} call ``stable''), and constrained efficient(pareto efficient among the priority-respecting matchings). Meanwhile, there is not way for the student to manipulate to get a better result for this example. 

This is true for the DA mechanism with quota.
Because of its importance, we state it as a formal theorem here.

\begin{thm}
A DA mechanism with quota is priority-respecting, constrained efficient and non-manipulable.
\end{thm}

The proof is relegated to the Appendices.

\subsection{Policy suggestions}

For student-college matching mechanisms, when no affirmative actions are taken, and the scores in the nationwide college entrance examination is the base priority criterion, then the achievable result under perfect
manipulation is the same, which is efficient, stable and
unique. However, only for the DA or score-based serial dictatorship,
the perfect manipulation is easily fulfilled in reality, since the
perfect manipulation is just non-manipulation, i.e., reporting the
true preference. Therefore, it's very easy to teach how to do a
successful manipulation in a score-based serial dictatorship, and even
a researcher cannot tell how to do a successful manipulation in
another mechanism such as IA mechanism and parallel mechanism. As a
result, a kind of unfairness caused by manipulation occurs. We do not
want to judge students by manipulation, but these IA and parallel
mechanism allow students' matching results to be influenced by a large
extent to such undesired manipulation abilities, especially for the
majority of students
who are not the top students. 

Where there is manipulation, there is
corruption. A successful manipulation(reporting of preference) in IA
and parallel mechanism needs information about other students'
preference reporting. So there might be buying and selling of such
information and even without corruption of relevant institutions, some
other unnecessary  businesses, like strategy
consulting of preference  reporting  for  the college admission
mechanisms.  These things  burden the family
of senior high school students unnecessarily both economically and
spiritually.




 