% Chapter 4

\chapter{Matching}  % Main chapter title

\label{Chapter4} % For referencing the chapter elsewhere, use \ref{Chapter4} 

%----------------------------------------------------------------------------------------

% Define some commands to keep the formatting separated from the content 
%\newcommand{\keyword}[1]{\textbf{#1}}
%\newcommand{\tabhead}[1]{\textbf{#1}}
%\newcommand{\code}[1]{\texttt{#1}}
%\newcommand{\file}[1]{\texttt{\bfseries#1}}
%\newcommand{\option}[1]{\texttt{\itshape#1}}


\section{Introduction of the basic concepts}



%I would like to introduce Coq Proof Assistant for this purpose, 
% for there are many theorems in matching theory which can be proved
% more strictly and without errors with its help. My motivation comes
% from the fact that proof error is easy to creep into human proofs 

% and most proofs can only be best understood and checked by the original author when the body of theorems and propositions in matching theory become more and more abstract in quality and large in
% quantity. 
 
% The adoption of Coq for this subject is an idea inspired by the 2002 Fields Medal Winner Vladimir Voevodsky's recent project Univalent foundations for mathematics. He uses Coq to formalize the proofs of theorems
% in homotopy theory and try to build another foundation of mathematics which can be viewed as equal to the set theory but is much easier for the computer to check the correctness of proofs. 
 
% first, we consider the easy one. Man woman marriage market as defined in the classical work of (Alvin Roth and SotoMayor). 
% We define these markets as a type with m men and n woman as follows
% Inductive Market (m n: nat) :=
% |mkt: Men n -> Women m -> Market m n
We consider two-sided matching in this chapter. There was a monograph dedicated to it, see \parencite{Roth1990}. However, we will focus on application of the theory to Chinese college admission mechanism, and use some notions that have recently developed. In this section, some of the most important theoretical concepts and results are introduced. 

In two sided matching, the two
distinguishable sides of matching are assumed to be men and women,
students and schools, workers and employers, tenants and houses,
etc. Here, students and colleges are chosen as the economic
background. For this specific topic, there
is \parencite{Sonmez2003}  which give a description of the literatures
in this field till that time. 

Now A description of the model and relevant notations that will be used in this chapter. There is a finite set of students and a finite set of colleges denoted by  $S = \{1,...,n\}$ and $C = \{1,...,m\}$, respectively. For any $j \in C$, college $j$ has $q_j$ quota, which is college $j$'s seats for students. Students wish to enter at most one college and have an option not to enter any college at all. This outside option is formally represented by a null college, denoted by $0$. This null college has unlimited quota, ie, $q_0 = + \infty$. 

A matching(sometimes also called assignment or allocation) is a
mapping $\mu : S \rightarrow C\cup\{0\}$ such that for any $j \in C$,
$|\mu^{-1}(j)| \leq q_j$. Denote by $\mu(0)$ the set of colleges with
seats that is not assigned to any student at matching $\mu$. The null
college is always included in this set, as its seat supply is
unlimited. Hence:
\[ \mu(0) = \{j \in C : |\mu^{-1}(j)| < q_j\}\cup \{0\}\]


We assume that students have preferences on colleges.  Denote by $R_i$ student $i$'s preference on the set of colleges $C \cup \{0\}$. The corresponding strict preference and indifference relations are denoted by $P_i$ and $I_i$, respectively. The meaning of the notation is as follows: if $c,c'\in C \cup \{0\}$ and $c R_i c'$, then student $i \in S$ weakly prefers college $c$ to college $c'$. Change ``weakly prefers'' to ``strictly prefer'' we get the meaning for the strict $P_i$. Change ``weakly prefers'' to ``are indifferent'' we get the meaning for the indifferent $I_i$. Preferences are assumed to be rational in the sense that for all $i \in S$, $R_i$ is complete, reflexive and transitive. A preference profile is a list $ R = (R_1,...,R_n)$ of the students' preferences. All the discussions here and after are based on the strict preference assumption. That is , if $c I_i c'$, then $c=c'$.\footnote{Because the students have strict preference, a college can only be indifferent to itself in any student's preference. No two things are exactly the same, and indifference are usually the result of inadequate information, so here we assume aspects of the university are known to the extent that the colleges can be ranked by a student.}

 Here, a priority structure $\pi$ is assumed for the colleges, that is, for each college $j \in C$, there is an exogenously given strict priority-order $\pi_j$. Formally, $\pi_j: S \rightarrow \{1,\cdots,n\}$ is a bijection where the highest-ranked student  is the student $i\in S$ with $\pi_j(i)=1$, the second highest ranked student $i' \in S$ has $\pi_j(i')=2$, and so on. When $\pi_j$ is the same for all $j\in C$, we get the serial dictator matching mechanism. A priority structure is a list $ \pi = (\pi_1,...,\pi_m)$ of the colleges' priority-order. For the concept of pairwise stable, this priority structure is enough. \footnote{For group stable or core, further definition of college preference on groups of students should be provided. And for substitutable college preferences which is an implied assumption of this chapter, the two concepts coincide. See Proposition 6.4 in \parencite{Roth1990} for more details.}

 A matching $\mu$ is pairwise stable(or called priority respecting as in \parencite{Svensson2014}) if there is no student $i \in S$ who strictly prefers some college $j$ to $\mu(i)$ and $\mu^{-1}(j)$ contains some other student $i' \in S$ who has lower priority for college $j$ than student $i$(i.e., $\pi_j(i')>\pi_j(i)$), and furthermore, all students weakly prefer their assigned college seats to any unassigned seats in $\mu(0)$(seats from the same college are obviously equal). We'll simply use stable to mean pairwise stable later in the chapter.

 Formally ,the definition of stableness is as follows.

\begin{definition*}(stableness)
A matching $\mu$ is stable for a given preference profile $R$ and a given priority structure $\pi$, if :

(i) for all $i,i' \in S$, $\mu(i')P_i \mu(i)$ only if $\pi_{\mu(i')}(i') < \pi_{\mu(i')}(i)$;

(ii)for all $i \in S$, $\mu(i)R_i j$ if $j \in \mu(0)$, where $\mu(0) = \{j \in C : |\mu^{-1}(j)| < q_j\}\cup \{0\}$.

\end{definition*}

Condition (i) is called fairness condition, with the meaning that
there is no justified envy\footnote{justified envy means that you envy somebody
else's college and according to your priority in that school, you
should be admitted prior to that person}. Condition (ii) is the combination of
individual rationality and non-wastefulness. Rationality means that
the current assignment of college for a student must be weakly better than the outside
option of null college 0. Non-wastefulness means that the current
assignment of college for a student  must be weakly better than any
unocuppied college seat. For more detailed discussion of these concepts
, see also \parencite{Sonmez1999}. 

The next important concept is Pareto efficiency for students or simply
efficiency. A matching is pareto efficient if no other matching can
make at least one student get a strictly more preferred college while
no student get a strictly less preferred college. 

Formally, we have the following definition.
\begin{definition}
A matching $\mu$ is efficient for a given preference profile $R$, if
for all $\mu'$ such that there exists $i \in S$ satisfying $\mu'(i)
P_i \mu(i)$, then there must be some other $i' in S$ satisfying
$\mu(i') P_{i'} \mu'(i')$
\end{definition}

It is clear that efficiency for students does not take into account of college priority while stableness does need it. Stableness is priority respecting and includes a flavor of fairness in it. Now that we have some idea about matching, we begin to investigate mechanism that can lead to stable and efficient matching. An important difference between matching mechanism design and other mechanism design such as auction is that efficiency is usually not the only requirement, stableness or priority respecting is at least of equal importance. We will illustrate this later in examples.

Why do we care about priority? In the previous chapter, we only care about efficiency. which means to maximize the sum of all participants' utilities. However, there is a well known saying that the whole is more than the parts put together. Priority should belong to the more-than part of the whole. suppose that only students entering their most preferred college will study hard, and any student would feel the same level of happiness when they enter their most preferred college but those with high scores will contribute more to the society, so high scored students should have priority. However just by the limited maximizing sum efficiency criterion we would send arbitrary student to their most preferred college, while from the society as a whole giving the high scored students high priority is a better choice.\footnote{I never believe this myself, so it is only a possible reason coined to show that priority may matter to the social welfare; other possibility of it may be fairness itself is an component of social happiness.}

When it comes to mechanism design, the usual problem of unilaterally misreporting one's private information(here is one's preference) emerges. In the literature, a mechanism that is immune to such problem is often called strategyproof. Viewing the reporting of one's preference under a mechanism as a game among the students, strategyproofness  means that it is a dominant strategy to report one's true preference. 

There is also a related concept called group-strategyproof. What is the connotation of this concept? It means that no group can get a pareto improvement by unilaterally changing the reports of students in this group under the mechanism. Viewing the reporting of one's preference under a mechanism as a game among the students, group-strategyproofness means that it is a core equilibrium for everyone to report the private true preference.



\section{Popular matching mechanisms}
\label{equivalent direct mechanisms}
There are many matching mechanisms. Three kinds are
most studied. They are the Deferred Acceptance Mechanism(DA), the Top
Trading Cycle Mechanism(TTC), and the serial dictatorship mechanism(SD).
These three mechanism are all related to the later analyzed Chinese
student-college matching mechanism, especial the DA mechanism. Why are they most popular? 
One reason is that all of these mechanisms are
strategyproof. Therefore the reported preference profile are true preferences
for rational agents.

Among these, DA mechanism is what will be of key importance in the later modelling and discussion of Chinese college admission
mechanisms. Thus an indepth investigation is provided in the following subsection.



\subsection{Deferred Acceptance Mechanism}
In the seminal work \parencite{Gale1962}, DA mechanism is first
proposed. This mechanism is the first and probably the most studied
mechanism in modern matching theory. 

In the setting of student-college matching, it is also called
student-optimal stable mechanism (SOSM) for it always finds the stable
matching that is most favorable to each student. Its outcome can be
calculated via the following Deferred
Acceptance (DA) algorithm for a given problem:

 Step 1: Each student applies to his or her favorite school. For each
college $j$, up to $q_j$  applicants who have
the highest college $j$ priority are tentatively assigned to college $j$. The remaining applicants are
rejected.

Step k ($ k \geq 2$ ): Each student rejected from a college at step k -1 applies to her next favorite college.
For each college $j$, up to $q_j$  students who have the highest college
$j$ priority among the new applicants and those tentatively on hold from an earlier step, are
tentatively assigned to college $j$. 
The remaining applicants are rejected.

The algorithm terminates when no student applies to any new college in
some step. That is, every  student is either tentatively placed to a college
or has been rejected by every college that is better than null college
in his or her preference list. 

We list important properties concerning the DA mechanism here.

Under strict priority( that is, no two colleges have  the same utility
level for a student.), the following properties
hold. \parencite{Gale1962} first proposed the two theorems and proved
them. The proofs for the theorems are short and elegant, and therefore
provided in Appendix \ref{Appendix_D}.

\begin{thm*}(Gale-Shapley)
The matching given by DA mechanism is stable.
\end{thm*}

\begin{thm*}(Gale-Shapley)
Every student is at least as well off under the assignment given by
the DA mechanism as he would be under any other stable assignment.
\end{thm*}

These two theorems can be combined into one concise statement. The SOSM
matching produced by the DA mechanism is the optimal stable matching.
An alternative way to express this is that if you want to find a
matching that is stable, the SOSM matching selected by the DA  mechanism is the most
efficient matching(constrained efficiency).

\begin{thm*}
\label{thm1}
The DA mechanism is strategy-proof for the students.
\end{thm*}

The proof of it is complex and need more notions, please see \parencite{Roth1982a} or \parencite{Roth1990}for proof.
However, the DA mechanism is not pareto efficient.  This fact can be shown by an example.

\begin{example}
There are two colleges $A$ and $B$, each with 1 seat for students. A null college $0$ with unlimited seats and accepts who ever applies for it. There are
3 students 1,2,3. The preferences of the students are
as follows.

    \begin{center}
      \begin{tabular}{|c|c|c|}
        \hline
        $1$ & $2$ & $3$ \\
        \hline
        $A$ & $B$ & $A$ \\
        
        $B$ & $A$ & $B$ \\

        $0$ & $0$ & $0$ \\

        \hline
        
      \end{tabular}
    \end{center}

The two colleges' base priority structure is the same as in the following
table

 \begin{center}
      \begin{tabular}{|c|c|}
        \hline
         $A$ & $B$ \\
        \hline
        $2$ & $1$\\
        
        $3$ & $3$ \\

        $1$ & $2$ \\
       
        \hline
        
      \end{tabular}
    \end{center}
 
The DA mechanism is run round by round as in the table below.

 \begin{center}
      \begin{tabular}{|c|c|c|c|}
        \hline
        &$1$ & $2$ & $3$\\
        \hline
       round1&   & $B$ & $A$ \\
        
        round2&$B$ &  & $A$ \\

        round3&$B$ & $A$  & \\

        round4&$B$ & $A$ & $0$\\
        \hline
        
      \end{tabular}
    \end{center}

Because 3 is the only student not tentatively accepted after 3 rounds,
and 3 has been rejected by both A and B, the DA algorithm with quota
terminates with student 1 entering B, 2 entering A, and 3 entering null college $0$.

Now consider the following alternative preference profile.
\begin{center}
  \begin{tabular}{|c|c|c|}
    \hline
    $1$ & $2$ & $3$ \\
    \hline
    $A$ & $B$ & $0$ \\
    
    $B$ & $A$ & $A$ \\

    $0$ & $0$ & $B$ \\

    \hline
    
  \end{tabular}
\end{center}

With this profile, the DA mechanism is run as below.

\begin{center}
  \begin{tabular}{|c|c|c|c|}
    \hline
    &$1$ & $2$ & $3$\\
    \hline
    round1& $A$  & $B$ & $0$ \\
    
    \hline
    
  \end{tabular}
\end{center}

Just 1 round, and the DA process is terminated.





\end{example} 

\begin{remark}
 If a mechanism is able to produce every possible matching results, then not pareto efficient obviously means that the mechanism is not group-strategyproof, since reporting the profile that can result in a pareto improvement assignment is a profitable group-deviation.
\end{remark}


%\subsection{Top Trading Cycle Mechanism}
%This mechanism is given in \parencite{shapley1974}. It is also an
%important mechanism and being studied and modified by scholars across
%the matching area since 1974 

%In the student-college matching setting, the TTC matching result is
%determined after the following process.


%\subsection{Serial Dictatorship Mechanism}

\section{Analysis of Chinese college admission mechanisms}

As an application of the matching theory, we will focus
on the Chinese college admission mechanisms in this section.

The Chinese college admission is a centralized matching mechanism with
an entrance exam taken national wide as a means of student-priority
deciding tool for the colleges.
Students from high school in China need to take an entrance exam to
decide the student priority for colleges, and the priority structure
is also influenced by the application form that the students write to
inform authority their own preferences. The college entrance exam, or Gaokao in Chinese Pinyin,  forms the 
foundation of the priority determination process for the current
admission system. However,the stage of priority determination through examimations is not the concern of us. There is a complication in the real system, the colleges are classified into batches, where each batch of colleges share the same enrollment scoreline, that is, the minimum score that a student must have in order to be considered eligible for colleges of that batch. However, Shanghai has first abandoned the classification of colleges into batches with different enrollment scoreline from 2016, and many other provinces have anounced their plan to reduce batches or simply abandon batches since then. This complication is not considered in our analysis.

We are interested in mechanisms about how to match the college and students once the priority has been determined. Starting from the early 1950s,
the mechanisms adopted for the stage of matching has gone
through series of reforms,  and what kind of mechanism should be adopted is
currently still in debate. There are millions of high school seniors compete
for  seats at various colleges and universities in China each
year. The matching of students to these colleges and universities has
profound implications for the education and labor market outcomes for
these students.  The fairness and efficiency of this matching process,
is of great importance to the society.

Many Chinese scholars have writen articles discussing problems related
to Chinese college admission mechanisms. \parencite{Zhong2004} has investigated the effects
of three preference report timings on the efficiency of the final matching, i.e. reporting preference
before the exam, after the exam but not knowing the exact scores,  and
after the scores are known.  This study has focused on the
strategical interaction of students in the games, not on how to induce
the truthful reporting with a suitable matching mechanism later in the match stage.  As to the strategical importance of
preference reporting,  Haifeng Nie has written many articles.  Both \parencite{Nie2007a}
and \parencite{Nie2007b} has stressed the point that a good score may
not be as useful as a good strategy for reporting the preference of
the university. This somewhat strange phenomenon has motivated us
in studying the issue of reducing manipulation and inducing better
and fairer results.

The Chinese  matching mechanisms has also recently caught attention of overseas scholars from
the economic field of matching. \parencite{YanChenJPE} has
done much to analyze theoretically the sutleties of the different
mechanisms adopted, and has gone even further to design experiments
for studying features of these mechanisms,
see \parencite{YanChen2016}. Though much has been investigated by them, we still find something interesting in addition to what they find.



\subsection{A simple characterization of the popular matching mechanisms in CCA}

The mechanisms are hard to analyze with all the real application-rejection procedures mixed in the definition,
so we will deal with the equivalent direct mechanisms of the mechanisms actually in use.

When studying ex post equilibrium results for existing
mechanisms which are not direct revelation mechanism, we will convert it to aa
equivalent direct revelation mechanism with the same matching results for expositional convenience\footnote{We put the
  mechanisms in the same revelation mechanism class. and this make it easier to do
  comparison.}  Also to aid analysis and exposition, a notion of successful
manipulation which means a best report choice given others' reports is
used in later analysis of the strategical interaction of
students in the CCA matching mechanisms. \footnote{ It is
  essentially the best response concept in Nash equilibrium but sounds
  more fluent when used in description of the strategical interactions
  among the students.}

Now, let's introduce three important class of mechanisms: DA mechanism, IA\footnote{Immediate Acceptance mechanism, also known abroad as the Boston Mechanism for its adoption in Boston district, uses the same accept-reject procedure as DA, only the acceptance is not tentative but final.}
mechanism and parallel mechanisms. DA mechanism is what we have investigated in the previous section, but it has never been truely adopted in China
due to the fact that the length of reported preference list is too small a number compared to the hundreds of colleges. The Immediate Acceptance (IA)
mechanism  had been the main mechanism used in Chinese
student assignments both at the high school and college
level \parencite{Nie2007b}. However, this mechanism has a serious
limitation: “a good score in the college entrance exam is often worth less
than a good preference reporting strategy” \parencite{Nie2007a}. In our paper's wording, a successful
manipulation is often more important than your score level. This
problem arises from the special priority structure in the
mechanism. Priority of a student in a college is based on the
reported preference list firstly and based on your scores secondly, a
dictionary order that put too much emphasis on preference
reporting. Given in \parencite{Nie2007b}, and also cited
in \parencite{YanChenJPE}, one parent has the following explanation of
the problem.

``My child has been among the best students in his school and school district. He
achieved a score of 632 in the college entrance exam last year. Unfortunately, he was
not accepted by his first choice. After his first choice rejected him, his second and third
choices were already full. My child had no choice but to repeat his senior year.''

To alleviate this problem of high-scoring students not being accepted by any universities, the
parallel mechanism was proposed. 
In the parallel mechanism, students select several
“parallel” colleges within each choice-band. The priority structure is changed. A detailed description from \parencite{YanChenJPE} is cited here for your full understanding of the mechanism.

``a student’s first choice-band may
contain a set of three colleges, A, B, and C while her second choice-band may contain another set
of three colleges, D, E, and F (in decreasing desirability within each band). Once students submit
their choices, colleges process the student applications, using a mechanism where students gain
priority for colleges they have listed in their first band over other students who have listed the same
college in the second band. Assignments for parallel colleges listed in the same band are considered
temporary until all choices of that band have been considered. Thus, this mechanism lies between
the IA mechanism, where every choice is final, and the DA mechanism, where every choice is
temporary until all seats are filled.''

In 2003, Hunan fisrt implemented the parallel mechanism, allowing 3
parallel colleges in the first choice-band(or called group), 5 in the second choice-band, 5 in the third choice-band, 5 in the fourth choice-band, and so on, see \parencite{YanChenJPE}.  The parallel mechanism is soon to be widely perceived as having improved allocation outcomes for students. Citing a parent from a newpaper report \footnote{Li Li. “Ten More Provinces Switch to Parallel College Admissions Mechanism This Year.” Beijing Evening News,
June 8, 2009.},

``My child really wanted to go to Tsinghua University. However,  in order not to
take any risks, we unwillingly listed a less prestigious university as her first choice.
Had Beijing allowed parallel colleges in the first choice band, we could at least give
Tsinghua a try.''

%However, I would like to describe all the three mechanisms
 %as generalized DA mechanisms with different priority structures.
In the paper \parencite{YanChenJPE}, they have formulated the IA
mechanism, parallel mechanisms, and DA mechanism as a parametric family
of application-rejection mechanisms. In this chapter, they are all
converted to their equivalent direct mechanisms. The following are the definitions.

\begin{definition*}direct revelation DA mechanism:
  
Each student reports his or her preference list, each college reports its priority criteria(usually scores plus some tie-breaking rules),then after collecting all these preference lists and priority criteria, the social planner runs the DA procedure
to produce a final matching result.
\end{definition*}

\begin{definition*}direct revelation IA mechanism:

Each students reports his or her preference list, each college reports its priority
criteria(usually scores plus some tie-breaking rules),then after
collecting all these preference lists and priority criteria, the
social planner adjusts the priority rule such that the reported
priority criterion of a college is made only a tie-breaking rule, and the rank of a college in a student's reported preference list is
the main priority criterion. Finally,
the social planner runs the DA procedure
to produce a final matching result.
\end{definition*}

The parallel mechanisms are a wide group of mechanisms. Therefore we need to devise a group of equivalent direct mechanism for them all. However,
parallel mechanisms only differ in the choice-band sizes, so different list of choice-band sizes characterizes a different parallel mechanism.

\begin{definition*} characteristic list of a parallel mechanism:
  
  An infinite list of integers greater than zero $(e_1,e_2,\cdots)$ is the characteristic list of a parallel mechanism if the $e_1,e_2,\cdots$
  define the choice-band sizes for the 1st ranked, 2nd ranked, $\cdots$  choice-bands.
\end{definition*}

Henceforth, a parallel mechanism with a characteristic list $(e_1,e_2,\cdots)$ will just be called mechanism $(e_1,e_2,\cdots)$.


\begin{definition*}direct revelation parallel mechanism $(e_1,e_2,\cdots)$:

  Each student reports his or her preference list, each college reports its priority
  criteria(usually scores plus some tie-breaking rules). After
  collecting all these preference lists and priority criteria, the
  social planner uses the preference list to produce a list of college
  choice-bands according to the characteristic list $(e_1,e_2,\cdots)$
  such that the 1st ranked choice-band
  includes the top $e_1$ colleges in the preference list, the 2nd
  ranked choice-band includes the top $e_2$ of the remaining colleges in the preference list, and so on. Then the social planner adjusts the
  priority rule such that the reported priority criteria of a college
  is made only a tie-breaking rule, and the rank of the choice-band
  which a college belongs to in a student's reported preference list
  is the main priority criterion. Finally, the social planner runs the
  DA procedure to produce a final matching result.
\end{definition*}

These three direct mechanisms are henceforth just called DA, IA and parallel mechanisms.
From the definition of parallel mechanisms, we can see that DA and IA mechanisms are two special forms of parallel mechanisms with characteristic
list $(+\infty,0,0,\cdots)$ and $(1,1,1,\cdots)$. Therefore they can also be called mechanism $(+\infty,0,0,\cdots)$ and mechanism $(1,1,1,\cdots)$.





The IA mechanism and parallel mechanism are qualitatively different from DA. Their
priority structure are closely related with the reported preference
while the priority structure of DA has no connection with the reported
preference. The underlying idea for the IA mechanism or parallel mechanisms is
``Colleges should like those who like them''.  This is exactly where manipulations creep
in.  

Formally, under the setting of student-college matching, we have the following definition.

\begin{definition}
A mechanism of matching is non-manipulable, if given any preference
$R$, and priority structure $\pi$, $\mu = h(R, \pi(R))$ and for any $i
\in S$, denote $h(R_i',R_{-i}, \pi(R_i',R_{-i}))$ by $\mu'$, then we have $\mu(i)R_i \mu'(i)$.
\end{definition}

We have the following claim.
IA and parallel mechanisms are manipulable. DA is non-manipulable.
Its proof can been found in \parencite{YanChenJPE}.
 
\begin{remark}
The efficiency and stableness concept is apparently not very useful when
the reported $R$ is not true. What we should be concerned is the status of the matching
result in the perspective of true $R$. So we have investigated the manipulability first.
\end{remark}

Because manipulation exists, investigating the matching result to the reported preference is of little meaning for the true social welfare. Therefore matching theory about the property of matching results to the reported preference is not applied in investigating Chinese college admission mechanisms. Without strategyproofness, we have to find another equilibrium solution concept other than dominant strategy equilibrium to study the likely matching results.

In this section we would like to view the mechanisms from a perspective of ex post Nash equilibrium.
For ex ante Nash equilibrium or Bayesian Nash equilibrium, there is a  best reponse concept.
A best response is a strategy for your type that maximize the ex ante payoff given the others' probability distribution on all probable types and each type's strategy choice. An ex ante Nash equilibrium or
Bayesian equilibrium is a strategy profile in which every player type chooses the best response. 
For ex post equilibrium,  a successful manipulation is similar in meaning to a best response to  ex ante Nash equilibrium. It means that given the other students' actual choice(action), the action you have
chosen is best to your interest. It deprives the intermediary probability distribution structure. This way we can concentrate on the final matching result.  Here, the ex post should not be taken 
literally, it does not mean that 
evaluating the action after some particular time. Actually, it means that assuming that every action of the other players have been known, the player still insists his choice due to its optimality with respect
to the others' actual action or choice. In fact, it may happen that the player will never know all the actions of the other players and therefore there is not a time point after which all information is revealed. 
The important thing is that the action or manipulation can be called successful when all information is known.
Formally, we give the following definition in the college admission setting. 
\begin{definition}
A student i has a successful manipulation 
if let $R$ denote the true preference, $R'$ the reported preference, $\mu' = h(R')$, then for any $R_i''$, let $\mu''= h(R_i'',R_{-i}')$, we have  $\mu'(i) R_i \mu''(i)$
\end{definition}

Here, I would like to explain this seemingly trivial concept. I think it is even more basic than ex post Nash Equilibrium. In fact, I extract it from the definition of ex post Nash equilibrium and find it more applicable than Nash equilibrium in my every day life. Its definition is in fact half of the Nash Equilibrium.

You may say that it is just the best response function. Almost, best response function is a way of getting a successful manipulation when the others' action choice is known. But calling it a successful manipulation is more appropriate in most cases as this concept does not assume that you know the payoff or the strategies of the others. Nash equilibrium usually assumes that the payoffs are known. But for a manipulation to be successful, it may not be a purposeful response but just happen to be successful and optimal. It is an ex post equilibrium but it may be possible that 
nobody knows the payoffs of the game fully.

A striking difference between Nash equilibrium and our ex post equilibrium which requires every player has a successful manipulation is that the Nash equilibrium is strongly repeatable but our ex post equilibrium is not since to be successful but not knowing it does not prevent you from trying something else the next time.


In this chapter, I consider deterministic game forms and only consider pure strategies. Our ``ex post'' concept requires that the strategies to be considered must be pure because you cannot report this preference and that preference at the same time when every thing is known\footnote{Only scholars who study quantum games may have disagreement. However,we do not consider quantum games here.}.

Why the concept of ex post equilibrium  deserves a special mention? Well, that is because in most real life situation the players may have chosen a successful manipulation but he or she does not necessarily know that he or she has achieved that and so can deviate from the successful manipulation. In that case, a disclosure of all the payoff functions to every player can make it a Nash equilibrium which is repeatable.   


 
When you have clear social criteria in mind, it is not good to devise a mechanism in which the social criteria are satisfied only when every agents have a successful manipulation. Not everyone 
is adapted at manipulation and has the information to achieve that. However, from the social point of view, the ex post equilibrium with every player successfully manipulated is still in a sense the fairest result that the mechanism can have. No body has done worse because of lack of informationin an ex post equilibrium.


We found that in the Chinese style college admission, the Boston mechanism, the parallel mechanisms and the DA mechanism have the same matching results in their expost equilibrium. However, it is not 
easy to achieve the expost equilibrium for the Boston mechanism and the parellel mechanism because they need lots of information to achieve it but the informtion is not easily attainable. On the contrary, the DA's ex post equilibrium can be achieved just by every player reporting the true preference, and it is indeed a dominant strategy equilibrium, stronger than merely an ex post equilibrium.

Now a definition that is useful in formally expressing the ideas in the previous paragraph.
\begin{definition}
under a preference reporting direct mechanism $h(.)$, students report some preference profile $R$, then the matching result is $\mu=h(R)$.
If all students have successful manipulation in this case, then the matching result $\mu$ is an ex post equilibrium result, and called achievable with perfect manipulation under the mechanism $h(.)$.
\end{definition}

There is a diffculty in analysing the true welfare property of a matching result achievable with perfect manipulation under a mechanism. The difficulty is that  even if you know that in the ex post equilibrium the matching result
is successfully manipulated by everyone, you do not know if it
satisfies a social goal $f(p)$, because you do not know the economics
environment $p$ because of manipulation of preference reporting!


In \parencite{YanChenJPE}, they started analysis of parallel mechanisms with the symmetric parameter setting, i.e., $e =e_1=e_2=\cdots$ before studying the general case where $e_1, e_2,\cdots$ might be independent parameters. In the general case, they proved a nested manipulability structure for parallel mechanisms, and then they paid attention to stability comparisons between parallel mechanisms with an additive decomposition relationship. For ex post equilibrium outcomes, they only give the Proposition 4 in the symmetric parameter setting. We found a nested structure of the ex post equilibrium outcomes in parallel mechanisms, and proved that the set of ex post equilibrium outcomes is not influenced by $e_2,e_3,\cdots$.   





 Before diving into the case of general priority structure, we first
focus our attention on a very important kind of priority structure,
the score-based priority structure and discuss what kind of matching
results can be achieved with perfect manipulation under such uniformly
score-based priority. This interest comes from reality. In CCA, the priority criteria is basically the scores with some similar tie-breaking rules across colleges, and the examination paper is usually the same nationwide\footnote{Only Beijing, Shanghai, and a few provinces choose to devise an exam paper of their own} .

DA mechanism just use the score high-low comparison to determine
priority of students for every college. The more score you get, the
high priority you get.

As to the adjusted priority structure in IA and parallel mechanisms, it is described as following.
To determine a student $i$'s priority in college $j$, parallel
mechanism fisrt group the colleges according to $i$'s reported
preference list. For example, in Tibetan's parallel mechanism, every
group has 10 colleges, then you group the first 10 together and label
every college in this Group 1, group the second 10 together and label
every college in this Group 2, and so on. A student's priority in a
college is determined in a dictionary order. First, compare the group
number of the college for the student. For instance, if Beijing
University is in the first 10 college group for a tibetan student, 
then this student has a higher priority in Beijing Unversity than
tthose who has not put  Beijing University in the first 10 college
group. If two students put the college in the same Group $n$, then
comes the second procedure of comparison. The student with higher
score(or high original priority) has higher priority. IA mechanism is
just a parallel mechanism with every group size equal to $1$.

In fact, the important thing is not that the priority structure is score-based nationwide, but that it is the same across colleges.
We have the following important theorem.

% and only modify the priority structure as described for the IA and parallel mechanism,
\begin{thm}\label{same}
If every college has the same base priority for the students, then the matching achievable with perfect
manipulation,that is, the ex post equilibrium matching result,  under the IA and parallel mechanism  is the same with the one under the DA mechanism, which is unique , stable and constrained efficient. 
\end{thm}

%In this section, we would like to show
%that the matching results that can be achieved with perfect
%manipulation is the same with the truthful revelation matching under the corresponding
%DA that results from deleting the adjustment that was done to the colleges' reported priority structure.\footnote{We do not consider college's
% strategic behaviour or preference, that is, we just view the report of colleges' priorities as a given fixed behaviour.}


%We have the following important theorem.

%\begin{thm}\label{same}
%If the colleges report the same priority criteria for the DA,IA and parallel mechanisms, then the matching achievable with perfect
%manipulation under the DA, IA and parallel mechanisms are the same  which is  unique ,  stable and constrained efficient. 
%\end{thm}







To get this result, we first propose a property of the DA mechanism which is convenient for discussion of its matching result.
\begin{definition}
  free-style DA:
  
  Actions includes:
  
  1.some students not tentatively holding by a college apply to their head college in their preference lists, and erase these colleges from the lists.
  
  2.some colleges with more than quota number of students applying rejects some students that are not in their top quota priority.

  Each round, action 1 happens or action 2 happens or action 1 and 2 happens together.
  
  
The termination condition for the free-style DA:

 When there is no college holding more than quota students, and no student is rejected by any college, the free-style DA ends.
 %that is his or her favorite among the colleges having not rejected him.
\end{definition}

The DA mechanism described before is a free-style DA such that each round all students not tentatively holding apply and all colleges with more than quota number of students reject all that are not in the top quota priority. The standard DA mechanism is the most efficient in operation among the free style DA mechanisms, but when the focus is on the final matching result, they are all the same DA mechanism. The following is a proposition for this.
\begin{prop}
  \label{irrelevant}
 The final
 matching result of free-style DA mechanism is  influenced by neither the number and order of applications of the students in each step nor the number and order of the rejection of colleges,  as long as there is some student who applies for a college in the preference list order  or a college rejects some student not in the top quota priority.
 
\end{prop}

\begin{proof}
  During the proof of the optimal stability of ordinary DA mechanism as shown in Appendix \ref{Appendix_D}, no specific order of application is assumed, therefore the order does not influence the final result's property of optimal stability. However, there is only one such result, hence the final matching result is the same.
\end{proof}

Since the order of application does not influence the final result, we can rearrange the DA application orders as we see convenient.
In fact, the DA mechanism produces the same matching result as  SD(Serial Dictatorship) mechanism when the base priority structure is the same. This can be seen easily by rearrange the DA application order by priority. The 1st prioritied student applies to his or her favorite college and is accepted ``finally'' because no later application can be from a student with higher priority. The 2nd prioritied student applies to his or her favorite college in the colleges with unfilled seats because he cannot compete with the previous applied students who have higher priority, and is accepted ``finally''. The 3rd, the 4th,... ,the nth, applies in the same manner.  Therefore, in the following discussion, we take the SD mechanism as equal to the DA mechanism under the environment of same priority structure. And we use terms of the DA matching result and the SD matching result interchangeably since they are the same in the same priority setting.

With this idea in mind, now an informal proof of theorem \ref{same} is given below. The proof is intuitionistic . The perfect manipulation of serial dictatorship mechanism is just non-manipulation, that is, stating the true preference. For the parallel mechanism, we would prove that for everyone to report as his or her favorite  the college he or she is matched to in the SD mechanism forms an ex post equilibrium. We only need to show that every student has got a successful manipulation by doing so. Using induction on the students' priority list(Note that we assumes strict priorities in this chapter).

First, the student $i$ with $\pi(i)=1$, anounce his or her true favorite college as his or her favorite, has successfully manipulated. He can obviously get  his or her true favorite this way. Moreover, this top student can only achieve a successful manipulation by reporting this way.

Second, for  an arbitrary student $i$. Suppose every student $i'$ such that $\pi(i')<\pi(i)$, i.e., higher in the priority list has reported as his or her favorite  the college that the SD mechanism assigns when he or she has reported truthfully. Now the student $i$ can only expect the SD resultas the best, and can indeed get it if reporting it as his or her favorite. Therefore reporting  as his or her favorite  the college that the SD mechanism assigns is a successful manipulation for him.

Having showed that the SD matching result is achievable with perfect manipulation, next we prove the uniqueness of the achievable matching result with perfect manipulation.

The proof of unique. Take any other matching result ,we would prove that it is not acheivable with perfect manipulation. Because the SD matching result is pareto-efficient, and the preference is strict, in any other matching result there must be some students who are worse off than in the SD matching result. There must be a student with highest priority among them, then we would like to show that this student has not successfully manipulated. Using case analysis: 

First, if this student is the top prioritied student, then he or she definitely has not successfully manipulated because if he or she report whatever college as his or her first choice, then he or she will get it.

Second, if  this student is not the top scored student, but all the higher prioritied students have successfully manipulated, then they will get the SD matching result, so this student can get the SD  matching result too. If he or she gets a worse college, he or she definitely has not successfully manipulated.

Except for the informal proof, later we will give another proof using the following theorem.
 

For general priority structures, as to matching results that can be
achieved with perfect manipulation under the IA mechanism, \emph{Theorem 1} in \parencite{Ergin2006} is of vital importance\footnote{They called them Nash equilibrium results which we do not think accurate as we have discussed the difference between Nash equilibrium and ex post equilibrium.}; as to matching
results that can be achieved with perfect manipulation under the
parallel mechanism,  \parencite{YanChenJPE} provided many deep insights.

\begin{thm}
  \label{expost-equilibrium}
  For two parallel mechanisms $(e_1,e_2,...)$ and $(e_1',e_2',...)$, if $e_1 < e_1'$, then any ex post equilibrium matching result in mechanisms $(e_1,e_2,...)$ can be implemented in ex post equilibrium with a properly chosen preference report in mechanism $(e_1',e_2',...)$.
\end{thm}

The concept of underdemanded school has been crucial in \parencite{Qianfeng2014}.
In the proof of the above proposition, a similar concept of underdesiredness is needed and defined below.
\begin{definition}
  A college $c$ is underdesired at a matching $\mu$ if no student $i$ strictly prefers $c$ to $\mu(i)$ relative to the student's true preference.
\end{definition}

Note that there is a difference between the underdesired concept here and the underdemanded concept in  \parencite{Qianfeng2014}. We say ``relative to the student's true preference'' while they mean ``relative to the reported preference''.

\begin{lemma}
  \label{yan}
  For the DA mechanism, a preference reporting profile $(R_1,...,R_n)$ gets the matching result $\mu$. For any student $i$, if he or she
  changes the preference report to just $\{\mu(i),...\}$,i.e.,put $\mu(i)$ at the first place, the resulting matching result under the
  new reported preference is $\mu'$. It must be
  the case that  $\mu'(i)=\mu(i)$.
\end{lemma}
\begin{proof}
  If we can show that deleting the report of the college before $\mu(i)$ in $R_i$ does not change $i$'s final matching result, then by repeated deleting the college before $\mu(i)$, we will move $\mu(i)$ to the first place while maintaining its final matching result unchanged.

  Suppose $R_i'$ is the preference which results from deleting the college $c$ before $\mu(i)$ in $R_i$ . Arrange the DA so that the student $i$ does not start the report of preference until the DA with all other students reporting has finalized. Then $i$ enters and report his preference list one by one. For $\mu(i)$ to be the final matching for $i$, all the college reports before it will fail\footnote{Downright rejection or first accepted then a rejection cycle is induced so that finally being rejected.}. There are two cases:
  
  First, if applying to the $c$ before $\mu(i)$ results in downright rejection, then deleting the application of it or not does not change the final matching of $i$.
  
  Second, if applying to  the $c$ before $\mu(i)$ results in a rejection circle, then: if the $\mu{i}$ is not involved in the circle, the final matching to $\mu{i}$ is not changed;if the $\mu(i)$ is a college in the circle, then $i$ can get in $\mu(i)$ after the circle has rotated means that before the circle rotates, it will initiate it and when it rotates back, make $\mu(i)$ reject the student that without $i$ would be tentatively accepted.
\end{proof}

Now is the formal proof of Theorem \ref{expost-equilibrium}.


\begin{proof}
  Suppose $R$ is an ex post equilibrium preference report in mechanism $(e_1,e_2,...)$, and produces the matching result $\mu$. At equilibrium, if a college $c$ has admitted $k$ students who have put it in any of the choice-band $e_2,...$, $c$ is an underdesired college. Also, $c$ must have more than $k$ seats unfilled after running  DA for the truncated preference lists where we only retain the first $e_1$ items in every student's preference list.
  %after the DA procedure of applications in the first choice-band $e_1$ of every student.
  Modify the preference report of $R$ to $R'$ as follows. If a student $i$ is matched to his or her first choice-band $e_1$, then $R_i'=R_i$. Otherwise, insert $\mu(i)$ to the end of choice-band $e_1$.

  First we claim that in mechanism $(e_1',e_2',...)$ this $R'$ produce the same matching result $\mu$. First, run  DA for the truncated preference lists where we only retain the first $e_1$ items in every student's preference list. This produces the same matching results for the students in the set $\{i| \mu(i) \text{ is in the first } e_1 \text{ colleges of } R\}$ since the $e_1$ items are in the $e_1'$ items of the $(e_1',e_2',...)$ due to $e_1<e_1'$. For any other student $i$, since they all insert $\mu(i)$ to the end of choice-band $e_1$, they are just matched to $\mu(i)$ too since $\mu(i)$ have enough seats unfilled. 

  Second we claim that $R'$ is an ex post equilibrium in mechanism $(e_1',e_2',...)$. Note that with the reported preference $R'$ each student is matched to a college in their first choice-band $e_1'$ in mechanism $(e_1',e_2',...)$, thus the running of the mechanism is equal to the DA. Suppose a student $i$ can become matched to a $\mu'(i)$ that is strictly preferred to $\mu(i)$ by $i$ relative to the student's true preference. Then by Lemma \ref{yan}, by moving $\mu(i)$ to the first of reported preference list, $i$ still gets $\mu'(i)$. Now consider $i$ report $\mu'(i)$ as 1st place under mechanism $(e_1,e_2,...)$, the preference reports of other students there can be viewed as essentially shorter than under mechanism $(e_1',e_2',...)$, so $i$ can get a matching result at least as good as $\mu'(i)$ under mechanism $(e_1,e_2,...)$ by \emph{Theorem 5.34} in \parencite{Roth1990}\footnote{ here it means that $i$ is matched to $\mu'(i)$ since $\mu'(i)$ is already the first in the reported list.}. This is contradictory to $R$ being an ex post equilibrium under mechanism $(e_1,e_2,...)$.


 
  
\end{proof}

An alternative proof of theorem \ref{same} is as follows.
\begin{proof}
  \emph{Theorem 1} in  \parencite{Ergin2006} tells that the ex post(or Nash) equilibrium results for IA(or Boston) mechanism are the same with stable matchings. The only ex post equilibrium matching result  in DA with same priority is just the optimal stable matching. Theorem \ref{expost-equilibrium} says that the ex post equilibrium results set is a subset of DA, but a superset of IA, and therefore contains only the unique stable matching in this priority structure.
\end{proof}  




For two parallel mechanism with the same first choice-band size, we have the following theorem.

\begin{thm}
  \label{first-equal}  
 For two parallel mechanisms $(e_1,e_2,...)$ and $(e_1',e_2',...)$, if $e_1 = e_1'$,then the set of all matching results from  ex post equilibriums  in mechanisms $(e_1,e_2,...)$ is the same with the set of all matching results from  ex post equilibriums in mechanisms $(e_1',e_2',...)$.
\end{thm}

\begin{proof}
  We only need to show that if $R$ is an ex post equilibrium of mechanism $(e_1,e_2,...)$ and produces the matching result $\mu$, then we can find a $R'$ such that $R'$ under mechanism $(e_1',e_2',...)$ produces the same matching result $\mu$ and reporting $R'$ is an ex post equilibrium under mechanism $(e_1',e_2',...)$.

  At equilibrium $R$, if a college $c$ has admitted $k$ students who have put it in any of the choice-band $e_2,...$, $c$ is an underdesired college. Also, $c$ must have more than $k$ seats unfilled after running  DA for the truncated preference lists where we only retain the first $e_1$ items in every student's preference list.

  Modify the preference report of $R$ to $R'$ as follows. If a student $i$ is matched to his or her first choice-band $e_1$, then $R_i'=R_i$. Otherwise, insert $\mu(i)$ to the end of choice-band $e_1$.

  First we claim that in mechanism $(e_1',e_2',...)$ this $R'$ produce the same matching result $\mu$. First, run  DA for the truncated preference lists where we only retain the first $e_1$ items in every student's preference list. This produces the same matching results for the students in the set $\{i| \mu(i) \text{ is in the first } e_1 \text{ colleges of } R\}$.  For any other student $i$, since they all insert $\mu(i)$ to the end of choice-band $e_1$, they are just matched to $\mu(i)$ too since $\mu(i)$ have enough seats unfilled.

  Second we claim that $R'$ is an ex post equilibrium in mechanism $(e_1',e_2',...)$.  Suppose through reporting $R_i''$ a student $i$ can become matched to a $\mu'(i)$ that is strictly preferred to $\mu(i)$ by $i$ relative to the student's true preference.  This $\mu'(i)$ must not be underdesired college. By Lemma \ref{yan}, by moving $\mu(i)$ to the first of reported preference list, $i$ still gets $\mu'(i)$. Now consider $i$ reports $\mu'(i)$ as 1st place under mechanism $(e_1,e_2,...)$, due to $e_1=e_1'$, the preference reports $R_i$ of other students can be viewed as essentially the same as under mechanism $(e_1',e_2',...)$(2nd,3rd and ...  choice-bands preference do not matter due to low priority ), so $i$ can get the same matching result  $\mu'(i)$ under mechanism $(e_1,e_2,...)$. This is contradictory to $R$ being an ex post equilibrium under mechanism $(e_1,e_2,...)$.
\end{proof}




  

Now let us go to the next subsection where we put some more details on.

\subsection{Student-college matching with affirmative actions}

In this subsection, we would like to consider extending the previous
model to cover affirmative actions.
When there are bonus-score given to some student, if the bonus-score
is accepted in every college, then we can just take his or her total score as
original-score + bonus-score. The rest is the same as the serial
dictatorship case analyzed in the previous subsection. However, if
there is difference between the accepted bonus-score
among colleges, then the priority structure goes into the
general priority model.

In the following parts, we will focus on affirmative action of quota and reserve. Quota is 
a prevalent affirmative action policy in school choice limits the number of
admitted majority students to give minority students higher chances to attend their
desired schools. To circumvent the inefficiency caused by majority
quotas, \parencite{Hafalir2013} offered a different interpretation of the affirmative action policies based on
minority reserves. With minority reserves, schools give higher priority to
minority students up to the point that the minorities fill the reserves.
we would like to adopt the minority reserve definition
in \parencite{Hafalir2013} with a little extension. we will give every group(including the majority group) a reserve that gives a student of this group priority
over other groups whenever the reserve number has not been reached. This model can cover the CCA province reserves policy widely in use.
Using this definition of reserve, we give the following
example to compare DA and IA.

\begin{example}
There are two colleges $A$ and $B$, each with two seats for students. There are
two students groups $a$ and $b$. Students $1,2,3$ are in group $a$ and
students $4,5$ are in group $b$. The preferences of the students are
as follows.

\begin{center}
  \begin{tabular}{|c|c|c|c|c|}
    \hline
    $1$ & $2$ & $3$ & $4$ & $5$\\
    \hline
    $A$ & $A$ & $A$ & $B$ & $B$ \\
    
    $B$ & $B$ & $B$ & $A$ & $A$ \\
    \hline
    
  \end{tabular}
\end{center}

As can be seen, the students in group $a$ all prefer college $A$ while
the students in group $b$ all prefer college $B$. 

The two colleges' base priority structure is the same as in the following
table

\begin{center}
  \begin{tabular}{|c|c|}
    \hline
    $A$ & $B$ \\
    \hline
    $1$ & $1$\\
    
    $2$ & $2$ \\

    $3$ & $3$ \\

    $4$ & $4$ \\

    $5$ & $5$ \\
    \hline
    
  \end{tabular}
\end{center}
The quota structure
of A and B are both 1 seat for group $a$, 1 seat for group $b$. Now
the DA mechanism is run round by round as in the table below.

\begin{center}
  \begin{tabular}{|c|c|c|c|c|c|}
    \hline
    &$1$ & $2$ & $3$ & $4$ & $5$\\
    \hline
    round1& $A$ & $A$ &  & $B$ & $B$ \\
    
    round2&$A$ & $A$ & $B$ & $B$ &  \\

    round3&$A$ &   & $B$ & $B$ &  $A$\\

    round4&$A$ & $B$ &  & $B$ &  $A$ \\
    \hline
    
  \end{tabular}
\end{center}

Because 3 is the only student not tentatively accepted after 4 rounds,
and 3 has been rejected by both A and B, the DA algorithm with quota
terminates.

Now consider the IA mechanism. If the quota priority is first compared
and considered, then the result is the same as DA. This fact can be
easily verified and we omit it here.  If the reported preference
order is first considered, as the name of Instant Accept is
indicating, then the result is as the table below.

\begin{center}
      \begin{tabular}{|c|c|c|c|c|c|}
        \hline
        &$1$ & $2$ & $3$ & $4$ & $5$\\
        \hline
       round1& $A$ & $A$ &  & $B$ & $B$ \\
        
        \hline
        
      \end{tabular}
    \end{center}

\end{example}

From the above example, we see that the matching result of DA mechanism with quota is not pareto efficient for the students. In fact, the comparison of the tables' 
matching results shows that IA 
mechanism's result is a pareto improvement. Student $2$ and $5$ gets strictly better result in IA mechanism than in DA mechanism. Why the efficiency result of the score-based DA is lost. The reason is that the efficiency in previous subsection comes from the serial dictatorship nature of DA with score as the same priority among colleges. Now with quota included, the priority structure is changed and that property is lost in the process. However, the resulted matching for DA is still priority-respecting(as \parencite{Svensson2014} call ``stable''), and constrained efficient(pareto efficient among the priority-respecting matchings) . Meanwhile, there is not a way for the student to manipulate to get a better result in DA with quota.

Despite the truth-reporting result of IA weakly pareto dominates the DA result, we should notice that the truthful reporting strategy as in the example does not constitutes an ex post equilibrium. The student $3$ has not manipulated successfully, if he or she has reported college $B$ as favorite, $3$ will not ended with matching no college. What are the relationships of ex post equilibrium matching results among DA with reserve or quota, IA with reserve or quota and parallel mechanism with reserve or quota in general? That is one of the important issues we will deal with in the following parts.

We start the investigation from DA with quota. For the DA mechanism with majority quota, we have the following proposition.

\begin{prop}
A DA mechanism with majority quota is priority-respecting, constrained efficient and non-manipulable.
\end{prop}
\begin{proof}
  A DA mechanism with majority quota can be  converted into an equivalent DA as follows.

  Break each college into two colleges, one is the majority college with quota seats, the other is the minority college with remaining seats. The majority college with quota seats inherit the same priority structure, and the minority college with the remaining seats only accept minority students and preserve the priority of them. The preference list for minority is adjusted such that each original college is divided into two college with the minority college put before the majority college.  The preference list for majority is adjusted such that each original college is divided into two college with the majority college put before the minority college. Then run DA with these divided colleges and adjusted preferences, one get exactly the same result as running DA with majority quota as described in \parencite{Kojima2012}.

  From the above perspective, a DA mechanism with quota is just a DA mechanism essentially, and retains all the feature of a DA mechanism. Therefore it is priority-respecting, constrained efficient and non-manipulable.
\end{proof}

For the DA mechanism with minority reserve, the same assertion holds.

\begin{prop*}(Hafalir-Yenmez-Yildirim)
  A DA mechanism with minority reserve is priority-respecting, constrained efficient(optimal stable) and non-manipulable.
\end{prop*}

This is equivalent to the proposition 1 in \parencite{Hafalir2013}. 
In their proof, they also show that an equivalent way to implement the deferred acceptance
algorithm with minority reserves is first to create a new matching problem with no af-
firmative action and then to apply the original deferred acceptance algorithm to this
market.

A DA mechanism with minority reserve can be converted into an equivalent DA as follows.

  Break each college into two colleges, one is the majority college with quota seats, the other is the minority college with remaining seats. The majority college with quota seats inherit the same priority structure, and the minority college with the remaining seats raise the priority of minority students  to the front of the majority students while keeping the priority relationship in the group unchanged. The preference list for minority is adjusted such that each original college is divided into two college with the minority college put before the majority college.  The preference list for majority is adjusted such that each original college is divided into two college with the majority college put before the minority college. Then run DA with these divided colleges and adjusted preferences, one get exactly the same result as running DA with minority reserve as described in \parencite{Hafalir2013}.




Quota and reserve are quite distinct measures. Quota is to restrict a group such that it won't occupy more than  a number $q$ seats. Reserve is  to help a group such that as long as the $r$ seats for a college are not occupied fully by the group, then a member of the group only need to compete with members in the same group to enter the college.

From the previous description of quota and reserve, we see that quota does not help any group directly while reserve does help the first $r$ number of the protected students in each college directly because some of them may not be admitted to their college without the reserve help. Majority quota $q^M$ puts a natural reserve $r^m$ on minority because no majority will get into the reserve, and besides that, it is more demanding since even if $r^m$ is not full, majority can not be filled.


As is pointed out in \parencite{Hafalir2013}, the Pareto domination relationship in Theorem 1 of \parencite{Hafalir2013} holds if and only if it holds for the student-optimal stable matchings under the corresponding policies. To see this, note that for each affirmative action policy, the
student-optimal stable matching Pareto dominates any other stable matching. However, we think the equivalent one is more concise, and give it
here as a theorem.
\begin{thm*}(Hafalir-Yenmez-Yildirim)
  
the student-proposing de-
ferred acceptance algorithm with minority reserves Pareto dominates the algorithm
with majority quotas when $q^M + r^m= q$.
\end{thm*}
Here we would like to provide another informal proof of it, which is much shorter and may provide some insight.
\begin{proof}
Consider the equivalent DAs for the DA with majority quota $q^M$, and DA with minority reserve $r^m$.
  
By proposition \ref{irrelevant}, we can run the equivalent DA for the DA with minority reserve first to get its final result. Then let the minority colleges ejecting all majority students  and continue running the equivalent DA for the DA with majority quota $q^M$ until a termination reach.

The result of this free style DA is the same as the equivalent DA for DA with majority quota $q^M$, only it postpone the rejection of some majority students in the minority colleges until a matching result of the equivalent DA for the DA with minority reserve is reached.

The continuing of DA is bad for students as they are heading to colleges lower in the preference list. Therefore the student-proposing de-
ferred acceptance algorithm with minority reserves Pareto dominates the algorithm
with majority quotas when $q^M + r^m= q$.


\end{proof}


Though minority reserve was first proposed explicitly in theory in \parencite{Hafalir2013}. However, the reserve method was practically in use in China for years. In CCA, a college often reserves a fixed number of students seats for a particular province. Through the years, students in the provinces with reserves have mostly benefited from such policy. \footnote{ The author of this thesis himself has enjoyed the benefit of such reserves in CCA, though there is a minimum-score requirement for enjoying the reserve which is called the enrollment scoreline for the first batch of schools.  } 
  
When considering reserves for provinces, the nested structure of ex post equilibrium results for DA, IA and parallel mechanisms still hold. We have the following theorem. 

\begin{thm}
  \label{expost-equilibrium-with-reserve}
  For two parallel mechanisms $(e_1,e_2,...)$ and $(e_1',e_2',...)$ with reserves for  provinces, if $e_1 \leq e_1'$, then any ex post equilibrium matching result in mechanisms $(e_1,e_2,...)$ can be implemented in ex post equilibrium with a properly chosen preference report in mechanism $(e_1',e_2',...)$.
\end{thm}

\begin{proof}
  We only need to convert the parallel mechanisms with the same reserve into equivalent parallel mechanisms with no reserve, and show that the converted mechanisms produce the same results and their first choice-band size still have the same comparative relationship.
  After receiving the preferrence report of the students and the priority report of the colleges, break each college into the subcolleges each with the reserved number of seats and raise the priority of the target group of students\footnote{For example, students from shanghai.}  to the front of the other students while keeping the priority relationship in this group unchanged. The preference list for a student in each reserve group is adjusted such that each original college is replaced by the list of subcolleges with the reserve subcollege for the student being raised to the head. Then multiply the characteristic list of the original parallel mechanism by the number of subcolleges, we get the new characteristic list.

  After all the above conversions, run the converted parallel mechanism, we can get a matching result. Combine the subcolleges back into the original college with all the admitted students, we get a matching result for the students and colleges. It is obvious that this matching result is the same with the matching result from  original parallel mechanism. For any two converted mechanisms, their first choice-band size still have the same comparative relationship because they are multiplied by the same number.
  
\end{proof}

Quota also does not influence the nested structure of ex post equilibrium results for parallel mechanisms with similar reason. The style of proof in this section is intuitional and informal, but the ideas are hopefully conveyed. All in all, we find that the nested structure of ex post equilibrium matching results is not influence by the added details of quota or reserve.

 






\subsection{Policy suggestions}

For student-college matching mechanisms, when no affirmative actions are taken, and the scores in the nationwide college entrance examination is the base priority criterion, then the achievable result under perfect
manipulation is the same, which is efficient, stable and
unique. However, for the DA or score-based serial dictatorship,
the perfect manipulation is easily fulfilled in reality, since the
perfect manipulation is just non-manipulation, i.e., reporting the
true preference. Therefore, it's very easy to teach how to do a
successful manipulation in a score-based serial dictatorship, and even
a researcher cannot tell how to do a successful manipulation in
any other mechanism such as IA mechanism and parallel mechanism. As a
result, a kind of unfairness caused by manipulation occurs. We do not
want to judge students by manipulation, but these IA and parallel
mechanism allow students' matching results to be influenced by a large
extent to such undesired manipulation abilities, especially for the
majority of students
who are not the top students. 

Where there is manipulation, there is
corruption. A successful manipulation(reporting of preference) in IA
and parallel mechanism needs information about other students'
preference reporting. So there might be buying and selling of such
information. Even without corruption of relevant institutions, some
other unnecessary  businesses, like strategy
consulting of preference  reporting  for  the college admission
mechanisms will arise.  These things  burden the family
of senior high school students unnecessarily both economically and
spiritually.




 