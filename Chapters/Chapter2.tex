% Chapter 2

\chapter{Implementation theory and Nash Implementation  }  % Main chapter title

\label{Chapter2} % For referencing the chapter elsewhere, use \ref{Chapter2} 

%----------------------------------------------------------------------------------------

% Define some commands to keep the formatting separated from the content 
%\newcommand{\keyword}[1]{\textbf{#1}}
%\newcommand{\tabhead}[1]{\textbf{#1}}
%\newcommand{\code}[1]{\texttt{#1}}
%\newcommand{\file}[1]{\texttt{\bfseries#1}}
%\newcommand{\option}[1]{\texttt{\itshape#1}}


\section{Introduction}
For dominant strategy implementation, we have mentioned the important
contributions of Hurwicz, Gibbard, Satterthwaite. In the first section
of this chapter, their results will be listed. These are
important impossibility theorems about which we can not design. Like
in physics, we know that we can not produce perpetual motion machine,
then we put our efforts on resource consuming machines which are
everywhere in use today. Likely, we should understand these proved
negative theorems well first, then we can begin our search on
mechanisms that can be designed and of some good qualities while
avoiding time waste on mechanism design that has been proved impossible.


As we have mentioned in \ref{Chapter1}, the agents may have perfect
information regarding the economic environment $e$. In this case, Nash
Implementation is the solution concept most used.
If every agent $i$ knows his own $e_i$ and
knows the distribution of $e$, then this is the imperfect information
case which can be dealt with using Bayesian method.  This is not in the scope of this paper.

The economic environment $e$ in the following part and most economic literature is mostly the preference profile or utility profile, usually denoted $R$ or $u$.
\section{Important impossibility results}

A few important important impossibility results from the literature of mechanism design are listed in this section.

\begin{thm*}(Gibbard-Satterthwaite)
  \label{gibbard-satterthwaite}
If the outcome set $A$ has at least 3 alterna-
tives, a social choice rule which is strongly individually incentive compatible and
defined on a unrestricted preference domain is dictatorial.
\end{thm*}

\parencite{Gibbard1973} and \parencite{Satterthwaite1975} first proposed this important theorem for implementation theory. It is generally known as 
Gibbard Satterthwaite Impossibility theorem. This impossibility result is very important. It has led our research to restricted preference domain, like in matching where the outcome is assumed to be ranked only by a player's self matched object. For its proof, the original paper is recommended for reference.

Pareto efficiency is often a basic requirement of economic
mechanisms. However, in \parencite{Hurwicz1972}
Hurwics  shows that  the Pareto efficiency and the truthful revelation is
fundamentally inconsistent even for the class of neoclassical economic
environments.
\begin{thm*}
(Hurwicz Impossibility Theorem) For the neoclassical pri-
vate goods economies, there is no mechanism < M, h > that implements Pareto efficient
and individually rational allocations in dominant strategy. Consequently, any revelation
mechanism < M, h > that yields Pareto efficient and individually rational allocations is
not strongly individually incentive compatible. (Truth-telling about their preferences is not
Nash Equilibrium).
\end{thm*}
Since the proof is not too long. We provide
an adapted proof in Appendix \ref{Appendix_B}.


\section{Nash implementation}
\parencite{Maskin1999} proposed a monotonicity concept that is later
called Maskin monotonicity, which is a necessary condition for Nash implementation. This condition along with no veto power
constitutes a simple set of  sufficient conditions 
for full Nash Implementation. 

Now, let us see the definintion of Maskin monotonicity.

\begin{definition*}(Maskin)
A social choice rule $f: \mathscr{R} \rightarrow A$ satisfies Maskin
monotonicity provided that

$\forall a \in A, \forall R\ R' \in \mathscr{R},$, if $a \in f(R)$ and [
$\forall i \in \{1, \dots, n\} \forall b \in A  ,\  aR_i b \Rightarrow
aR'_i b$ ], then $ a \in f(R')$.
\end{definition*}

Quoting \parencite{Maskin1999}, ``In words, monotonicity requires that if alternative $a$ is $f$ optimal with respect to some
profile of preferences and the profile is then altered so that, in each individual’s ordering,
$a$ does not fall below any alternative that it was not below before, then $a$ remains $f$ 
optimal with respect to the new profile.'' The $f$ optimal in the above quotation means that the alternative is chosen by the social choice rule $f$.

To illustrate the concept, \parencite{Maskin1999} provided some examples of
mechanisms satisfying Maskin Monotonicity. As a first example, he considered the Pareto optimal correspondence $f^{PO}$. 
If $a$ is (weakly) Pareto optimal with respect to $R$, then for all $b$,  $\forall i \ a R_i b$. Now if we replace $R$ by $R'$ such that, for all $i$, $a R_i b \Rightarrow  a R'_i b$, we conclude that
for all $b$, $\forall i \ a R'_i b$. Hence, $a$ is also (weakly) Pareto optimal with respect to $R'$,  establishing the monotonicity of $f^{PO}$.

The Condorcet correspondence $f^{CON}$ is also Maskin monotonic. If $a$ is a majority winner
for a strict profile (a profile consisting of strict orderings) $R$, then, for any other alternative
$b$, the number of individuals preferring $a$ to $b$ is no less than the number preferring $b$ to
$a$. Formally, 
$ |\{i|a R_i b\}| \geq |\{i|b R_i a\}| $ where the $|$s deliminating a set stands for the number of elements in the set.
Now if $R'$ is a profile such that, for all $i$, $a R_i b \Rightarrow a R'_i b$, then the left-hand side of the inequality cannot
fall when we replace $R$ by $R'$. Furthermore, if the right-hand side of the inequality rises, then a contradiction happens since for strict relation 
$R\ R'$ $|\{i|a R_i b\}| + |\{i|b R_i a\}|= |\{i|a R'_i b\}| + |\{i|b R'_i a\}|= n $. Therefore we conclude that the inequality continues to hold when $R'$ replaces $R$, and
so $a$ is still a majority winner with respect to the profile $R'$, establishing the monotonicity of $f^{CON}$.


The following  is an important theorem proposed by \parencite{Maskin1999}.

\begin{thm*}
If $f: \mathscr{R} \rightarrow A$ is an SCR that is fully implementable in
Nash equilibrium, then it is Maskin monotonic.
\end{thm*}
The proof is provided in Appendix \ref{Appendix_B}.

\parencite{Maskin1999} proposes a No Veto Power(NVP) concept, together with
Maskin monotonicity will be sufficient to guarantee full Nash
implementability. Here is its definition.

\begin{definition*}(Maskin)
An social choice rule SCR $f:\mathscr{R} \rightarrow A$ satisfies NVP
if,
$\forall R \in \mathscr{R}$,$\forall a \in A$, and $\forall i \in
\{1,\dots, n\}$, 
($\forall j \not = i$ and $\forall b \in A$, $a R_j
b$) $\Rightarrow a \in f(R)$.
\end{definition*}

Now the famous theorem of Maskin:
\begin{thm*}
If $n\geq 3$ and $f: \mathscr{R}$ is a  n-person SCR satisfying Maskin
monotonicity and No Veto Power,  then it is implementable in Nash equilibrium.
\end{thm*}
For the proof,\parencite{Maskin1999} is recommended for reference.

The proof of it is illustrative of how to prove implementability. It
is constructive in nature. This proof method has been
in \parencite{Repullo90}.  \parencite{Maskin1999} adopted the same
approach. 


Now we have a sufficient condition. However, it is not a necessary
condition. Two examples here.

\begin{example*}(constant social choice rule)
A constant social choice rule . A social choice rule SCR is called
a constant  social choice rule if  there is a $ C \subset A$ such that
$ \forall R \in \mathscr{R}, f(R) = C$. For a constant social choice
function, a Nash implementation mechanism can be given by letting  people
report their preference, and then choose the constant social choice
$C$ regardless of what they report.  Obviously, what ever the report
is, there is not profitable deviation. Therefore, every report profile
constitutes a Nash equilibrium whose result is the $C$, i.e., $h(m)=f(R)=C$.
\end{example*}

\begin{example*}(dictatorial social choice rule)
A  dictatorial  choice  rule. A social choice rule SCR is called a
dictatorial choice rule if $\exists i \in \{1,\dots,n\} $such that
$\forall R \in \mathscr{R}\  f(R) = top(R_i)$(here $top(R_i)$ means
the highest ranked $a \in A$ according to $R_i$). For a dictatorial
social choice function, a Nash implementation mechanism can be given by letting  people
report their preference, and then choose the dictator's top ranked
choice according to his or her reported preference regardless of what
others report. Obviously, what ever the others report, if the dictator
report his or her true preference, the result is a Nash equilibrium
profile. Therefore, every report profile
constitutes a Nash equilibrium whose result is the $top(R_i)$, i.e., $h(m)=f(R)=top(R_i)$.
\end{example*}

The questions  are then:  what happens  in  the  grey  area  between
these  necessary and sufficient conditions of Nash implementation; and
what  happens  in the case  of two  agents? 


Actually, \parencite{Repullo90} has answered these questions.
\parencite{Danilov1992} provides another essentially monotone condition
that is  necessary and sufficient. \parencite{Yamato1992} extended the
\parencite{Danilov1992}  conditions for Nash implementation to weak
preferences over an arbitrary set of alternatives. Anyway,
these necessary  and sufficient  condition was not easy to identify,
and  \parencite{Maskin1999} proposed the easier to identify
sufficient condition and a separate necessary condition that are
well-known as Maskin monotonicity. \footnote{ Actually,  \parencite{Maskin1999} has been
widely known since 1977 as a working paper,  all the later papers
 had been written under the influence of it and all the search endeavor for necessary and
sufficient conditions have been in part motivated by Maskin's work.} In this retrospective
chapter, we also list these conditions below. They showed how important constructive proof method is
for mechanism design theory and are also listed here for the completeness of this line of literature. The direct results are not used in
the following chapters, so they can be safely skipped without hindering reading of the following chapters.

\subsection{Condition $\mu$}

First, let us take a look at the necessary and sufficient condition
in \parencite{Repullo90} which is the most general form of condition
. It is called condition $\mu$ which contains three parts.
\begin{definition*}(Moor and Repullo)
  Condition $\mu$: There is a set $B \subset A$, and $\forall i \in I,
  R\in\mathscr{R}, a \in f(R)$, there is a set $C_{i}(a, R) \subset
  B$, with $a \in M_{i}(C_{i}(a,R),R)$ such that $\forall R^* \in
  \mathscr{R}$, the following (i), (ii) and (iii) are satisfied.

(i) if $a \in \cap_{i \in I} M_i(C_i(a, R),R^*)$, then $ a \in f(R^*)$.

(ii)if $c \in M_i(C_i(a, R), R^*) \cap [ \cap_{j \not = i} M_j(B, R^*)]$, then $c \in f(R^*)$.

(iii)if $d \in \cap_{i\in I} M_i(B, R^*)$, then $d \in f(R*)$. 

\end{definition*}
In the above definition, $I= \{1,\dots,n\}$, and the $M_{i}$ has the following meaning. For
any $i \in I, R \in \mathscr{R}, C \subset A$, $M_{i}(C, R)$ denote
the set  of maximal  elements  in $C$ for  agent  $i$  under  preference $R_{i}$.

Now, the theorem in \parencite{Repullo90} is cited here.

\begin{thm*}(Moor and Repullo)
\label{mu}
  Suppose there are three or more agents. Then a choice rule $f$ is Nash implementable if and only if it satisfies Condition $\mu$.
\end{thm*}

A explanation of the condition $\mu$ is needed here. It is not as complex as one may feel at a quick glance. In fact, 
condition $\mu$ (i) is an equivalent of Maskin monotonicity.  To see
this, let $C_i(a,R)= L_i(a,R)$, then Maskin's monotonicity
$\Rightarrow$ condition $\mu$ (i). The other direction,  condition
$\mu$ (i) $\Rightarrow$ Maskin's monotonicity , is shown as follows.
Choose any  $R, R' \in \mathscr{R}$ and $a \in f(R)$ such that $L_i(a, R)
\subset L_i(a, R')$ for all $i \in I$. From  Condition $\mu$ we know
that  for  each $i \in I$ and $R \in \mathscr{R}$ there  exists  a set
$C_i(a, R)$ such  that  $a \in  M_i(Ci(a,  R),  R)$  implying  that
$C_i(a, R) \subset  L_i(a,R)$.   Hence $C_i(a, R) \subset  L_i(a,R')$,
which is equivalent to $a \in \cap_{i \in I}M_i(C_i(a,R),
R')$. Therefore by condition $\mu$ (i) we conclude that $a \in
f(R')$.  $f$ is thus shown to be monotonic.

Condition $\mu$ (ii) (iii) are implied by No Veto Power. Let $B=A$, and 
we can see this easily.

To see the usage of such a necessary and sufficient condition, a proof
of a proposition is provided in \parencite{Repullo90}.

First, they introduce a concept called neutral social choice rule.
\begin{definition*}(Moor and Repullo)   
Formally,  a choice rule  f  is  said  to  be neutral  if  for  all  permutations  $\pi:  A \rightarrow A$  and  $R \in \mathscr{R}$
  we have an $R^\pi \in \mathscr{R}$ such that $f(R^\pi) = \pi \circ f(R) $  and  $\forall i \in I: a R^\pi_i b \Longleftrightarrow \pi^{-1}(a) R_i \pi^{-1}(b)$.  
  \end{definition*}

That is, a neutral social choice function does not care about what the
allocation really is, it only cares about what the agents has ranked
these allocations and then decides. Apparently, a constant social
choice rule  is not neutral. However, a dictatorial social choice
rule is neutral as you can check according to the definition.


\begin{prop*}(Moor and Repullo)
  Suppose there are three or more agents. Then a choice rule f is
Nash implementable if it is monotonic and neutral.
  
\end{prop*}

\begin{proof}

  Now we will prove it by condition $\mu$ and theorem \ref{mu}. The proof is adapted from
  \parencite{Repullo90}.  

Monotonicity is equivalent to condition $\mu$ (i). Now we need to show
monotonicity and neutral imply
condition $\mu$ (ii)(iii).

For the implication of condition $\mu$ (ii). $\forall R, R' \in
\mathscr{R}, a \in f(R), i \in I$, if $c \in M_i(C_i(a, R), R') \cap [ \cap_{j
  \not = i} M_j(B, R')]$, we need to show $c \in f(R')$. Choose $R''
\in \mathscr{R}$ such that it is the same as $R$ except that outcomes
$a$ and $c$ are switched. By neutrality, $a \in f(R) \Rightarrow c \in
f(R'')$. By monotonicity, $c \in f(R'')  \Rightarrow c \in f(R')$ as
required.

For the implication of condition $\mu$ (iii). $\forall R' \in
\mathscr{R}$, if $ d \in \cap_{i \in I}M_i(B, R')$, we need to show $d
\in f(R')$. Choose $h \in f(R')$. If $h = d$, then the proof is done. 
If $h \not = d$, choose $\hat{R}
\in \mathscr{R}$ such that it is the same as $R'$ except that outcomes
$d$ and $h$ are switched. By neutrality, $h \in f(R') \Rightarrow d
\in f(\hat{R})$. By monotonicity, $d \in f(\hat{R}) \Rightarrow d \in
f(R')$ as required.
\end{proof}

\begin{thm*}(Moor and Repullo)
\label{mu}
  Suppose there are three or more agents. Then a choice rule $f$ is Nash implementable if and only if it satisfies Condition $\mu$.
\end{thm*}

\begin{proof}
Necessity.  There must be a range for the Nash implementing mechanism
$\Gamma = \langle S, g\rangle$. Let it be the $B$ in condition $\mu$.

\[ B\equiv \{a \in A|a=g(s)\ for\ some\ s \in S\}\]

 According to full Nash implementation, $\forall  R\in\mathscr{R}, a \in f(R)$, there is a Nash equilibrium profile $s$
 implementing the $a$, denote it by $s(a, R)$.  $\forall i \in I$,  we choose 
\[C_i(a,R)\equiv \{ c \in A | c = g(s'_i,s_{-i}(a,R))\  for\  some\  s'_i \in
  S_i\}\]
After finding the $B$ and $C_i$,  $\forall R^* \in \mathscr{R}$, condition $\mu$ (i)(ii)(iii)  are
directly implied by the full Nash implementation.

if $a \in \cap_{i \in I} M_i(C_i(a, R),R^*)$, then $s(a,R)$ is a Nash
equilibrium of the mechanism under $R^*$.  Since the mechanism fully
implement the social choice rule $f$, $a \in f(R^*)$.

if $c \in M_i(C_i(a, R), R^*) \cap [ \cap_{j \not = i} M_j(B, R^*)]$,
then $c = g(s'_i,s_{-i}(a,R))$ and is the best outcome among outcomes with the form
  $g(\hat{s_i},s_{-i}(a,R))\  where \  \hat{s_i} \in
  S_i$  under $R^*$, and $c$ is the best outcome for all $j \not = i$ in $B$ under
  $R^*$.  Therefore $(s'_i,s_{-i}(a,R))$  is a Nash
equilibrium of the mechanism under $R^*$.  Since the mechanism fully
implement the social choice rule $f$, $c \in f(R^*)$.

if $d \in \cap_{i\in I} M_i(B, R^*)$, then $d$ is the best outcome for
all $i \in I$ under $R^*$ and therefore the strategy profile $s(d,R*)$
producing $d$ is a Nash equilbrium of the mechanism under $R^*$. 
Since the mechanism fully
implement the social choice rule $f$, $d \in f(R^*)$. Thus, necessity
is  proved.

Sufficiency. For any  $f$ satisfying the condition $\mu$, it  is
enough to construct a mechanism $\Gamma$ which Nash
implements the social choice rule $f$  fully. The construction method of such
a mechanism is adapted
from the appendix of \parencite{Repullo90}. 

For each agent $i \in I$ , the message or strategy space is 
\[ S_i \equiv \mathscr{R} \times A \times Z^+ \]
and the outcome function $g: S \rightarrow A$ is defined as follows.

Case 1. If $\forall i \in I, s_i =  (R, a, n)$ such that $R \in
\mathscr{R}, a \in A, n \in Z^+$, then $g(s)=a$.

Case 2. If $\exists i \in I, \forall j \not = i \ s_j=(R, a, n) \  and
\ s_i = (R', a', n')$ where $ (R, a, n)  \not = (R', a', n')$, then 
$$g(s)= \begin{cases} 
 a'  & \text{if $a' \in C_i(a, R)$}\\ 
 a  & \text{otherwise} \end{cases}$$

Case 3. For all the other situations, $g(s) = a_i$, where $a_i$ is the agent with the largest
reported $n$'s choice of  $a$ (ties are broken by random selection).

Thus, the mechanism $\Gamma$ is fully defined.
We need to prove $\forall R in \mathscr{R} f(R) = NE(\Gamma, R)$. For
any $R \in \mathscr{R}$, the following two facts are showed to
complete the proof.

$f(R) \subset NE(\Gamma, R)$. $\forall a \in f(R)$, we consider the
strategy profile $s$ satisfying that $\forall i \in I, s_i = (R, a,
1)$. It is a Nash equilibrium, because  if one deviates, Case 2
stipulates that other than $a$ only an $a' \in C_i(a,R)$ can be chosen
which is worse than $a$ according to condition $\mu$($a \in
M_{i}(C_{i}(a,R),R)$ ).

$NE(\Gamma, R) \subset f(R)$.  There are three equilibrium
cases. 

Case1: $\forall i \in I, s_i =  (R', a, n)$ , then $g(s)= a \in
\cap_{i \in I}M_i(C_i(a,R'), R)$ because deviation can lead to any
element in $C_i(a,R')$ and Nash equilibrium should guarantee no
profitable deviation. By condition $\mu$ (i), $g(s) \in f(R)$.

Case2: $\exists i \in I, \forall j \not = i \ s_j=(R', a, n) \  and
\ s_i = (R'', a', n')$ where $ (R', a, n)  \not = (R'', a', n')$, then
$$g(s)= \begin{cases} 
 a'  & \text{if $a' \in C_i(a, R')$}\\ 
 a  & \text{otherwise} \end{cases} $$

and $g(s) \in  \in M_i(C_i(a, R'), R) \cap [ \cap_{j \not = i} M_j(B,
R)]$ because $i$'s deviation can lead to any element in $C_i(a, R')$
while $\forall j \not = i$  $j$'s deviation can lead to any element in $B$ by
anouncing a big enough $n$. By condition $\mu$ (ii), $g(s) \in f(R)$.


Case3: For all the other situations other than the previous two cases,
anyone can pronouce a big enough $n$ to get any  element in $B$. Nash
equilibrium means that the chosen $g(s)=a \in \cap_{i\in I} M_i(B,
R)$. By condition $\mu$ (iii), $g(s) \in f(R)$.

\end{proof}




%Formally,  a choice rule  f  is  said  to  be neutral  if  for  all  permutations  π :  A → A  and  θ ∈ Θ  we have θπ ∈ Θ and π·  f(θ)  %=f(θπ)-where  the  profile θπ is  such that  for  all  i ∈ I: aRi(θπ)b if and  only  if π -1(a)Ri(θ)π -1(b).  

%Condition $\mu$: There  is  a  set  B ⊂A,  and  for each  i ∈ I, θ ∈ Θ, and a∈f(θ), there  is  a set Ci(a,  θ) ⊂B, with 
 %    a∈Mi(Ci(a,  θ), θ )  such  that  for  all θ*∈ Θ,  the following (i),  (ii) and  (iii)  are satisfied. 
 % (i)if a∈∩ i ∈ IMi(Ci(a,  θ), θ* ) , then a∈f(θ*).
  %(ii)if c∈Mi(Ci(a,  θ), θ* ) ∩(∩j≠iMj(B, θ* )), then c∈f(θ*).
  %(iii)if d∈∩ i ∈ IMi(B, θ* ), then d∈f(θ*).
%For  any i∈I ,  θ∈Θ, and  C⊂A,  let  Mi (C, θ)  denote the set
%of maximal  elements  in C for  agent  i  under  θ.

\subsection{Essentially monotonic}
\parencite{Danilov1992}  proposed an essentially monotone concept
and \parencite{Yamato1992} extended its application conditions and
called it strongly monotonic. We will call it essentially  monotonic in this
paper. 

Some definitions are needed. First, the notion of essential element
for a participant in a particular set of outcomes.

\begin{definition*}
For any $i \in I$ and $X \subset A$, an alternative $x \in X$ is
essential for $i$ in set $X$ if $\exists R \in \mathscr{R}$ such that
$x \in f(R)$ and $ L_i(x, R) \subset X$.
\end{definition*}

The set of all essential elements for $i$ in $X$ under a social choice
rule $f$ is denoted as $Ess_i(X, f)$.

We can now define the notion of essential monotonicity.

\begin{definition*}
A social choice rule $f$ is essentially monotonic if $\forall R, R' \in
\mathscr{R}, a \in f(R)$,  $Ess_i(L_i(a, R), f)
\subset L_i(a, R') for all  i \in I$ implies $a \in f(R')$. 
\end{definition*}

A useful rewrite of this essentially monotonic condition is provided
in \parencite{Yamato1992}. That is, a social choice rule satisfies essential
monotonicity if :  $\forall R, R' \in \mathscr{R}, a \in f(R), a \not
\in f(R')$,  there exists $i \in I, b \in A$ such that (i) $a R_i b$
and $b P_i a$; (ii) $\exists \hat{R} \in \mathscr{R}$, $b \in
f(\hat{R})$ and $L_i(b, \hat{R}) \in L_i(b, R)$.
This rewrite has done a contrapositive to the definition and
insert in the definition of essential elements $Ess$. 

\begin{remark}
Some properties of essential elements and essential monotonicity are
listed here. 

For any $ B \subset C \subset A$,  $Ess_i(B,f) \subset Ess_i(C,f)
\subset Ess_i(C,f) \subset Ess_i(A,f) = Im(f)$. Essential monotonicity
implies Maskin monotonicity, because $Ess_i(L_i(a, R), f) \subset L_i(a,R)$.
\end{remark}


\parencite{Yamato1992} proposed a requirement of Condition $D$ on the preference
domain $\mathscr{R}$.

\begin{definition*}
The preference domain $\mathscr{R}$ satisfies Condition $D$ if
$\forall a \in A, r \in R, i \in I, b \in L_i(a, R)$, there exists $R'
\in \mathscr{R}$ such that $L_i(a,R) = L_i(b, R')$ and $\forall j \not
= i  L_j(b,R')=A$
\end{definition*}

Condition $D$ can be thought as a requirement that the preference
domain $\mathscr{R}$ should contain sufficiently many
preferences. For example, unrestricted domain therefore easily
satisfies condition $D$.

The following necessity theorem of Nash implementation is
from \parencite{Yamato1992}.

\begin{thm*}
If the preference domain $\mathscr{R}$ satisfies Condition $D$, and
a social choice rule $f :\mathscr{R} \rightarrow A$ is fully Nash
implementable, then $f$ is essentially monotonic.

\end{thm*}

\begin{proof}
Suppose mechanism $\Gamma=(S,g)$ Nash implements a  social choice rule
$f$, $R, R' \in \mathscr{R}$. Due to full implementation, $f(R)=
g(NE(R))$ and $f(R')= g(NE(R'))$. We then use  the contrapositive rewrite of
the essential monotonicity definition mentioned above to show that $f$
is essentially monotonic. For any $a \in f(R)$ and $a \not \in f(R')$,
there is a $ s \in NE(R)$ such that $a =g(s)$ ; there is a $i \in I, b
= g(s'_i, s_{-i})$ such that  $ b P'_i a$. Obviously  $b \in L_i(a,
R)$ by Nash equilibrium concept, so the (i) of the rewrite
of essential monotonicity is fulfilled. From condition $D$, $\exists
\tilde{R} \in \mathscr{R}$ such that $L_i(b, \tilde{R})=L_i(a,R)$
and $\forall j \not = i, L_j(b, \tilde{R})=A$. Since $g(S_i, s_{-i})
\subset L_i(a,R) $ and $\forall j\not = i, g(S_j, s_{-j}) \subset A$, therefore $g(S_i, s_{-i})
\subset L_i(b, \tilde{R}) $ and $\forall j \not = i, g(S_j, s_{-j})
\subset L_j(b, \tilde{R})$. $b \in NE(\tilde{R}) = f(\tilde{R})$.
\end{proof}

The sufficiency theorem of Nash implementation for more than 3
participants are proposed and proved in both \parencite{Danilov1992}
and \parencite{Yamato1992}. The following is the theorem.

\begin{thm*}(Danilov,Yamato)
If there are more than 3 participants, then an essentially monotonic
social choice rule  is fully Nash implementable.
\end{thm*}
The proof method is still by construction. Concrete steps are provided in \parencite{Danilov1992}
and \parencite{Yamato1992}.  An application of this essential monotonicity concept is provided
in \parencite{SonmezJet1996}.

