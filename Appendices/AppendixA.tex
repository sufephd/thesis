% Appendix A

\chapter{Appendix A: revelation principles} % Main appendix title

\label{AppendixA} % For referencing this appendix elsewhere, use \ref{AppendixA}
\begin{prop}
revelation principle in an interdependent value environment:

if a Mechanism $\langle M, h\rangle$ implements the social choice rule $F$ in ex post 
equilibrium. Then there is a direct revelation mechanism which implements $F$ truthfully in ex post equilibrium(truth telling is a ex
post equilibrium). 

\end{prop}
\begin{proof}
 Since $F$ can be implemented in ex post strategies by  $\langle M, h\rangle$, there is a profile of strategies $(\sigma^1,\cdots,
 \sigma^n)\in (S_1\mapsto M_1)\times \cdots\times (S_n\mapsto M_n)$ that forms an ex post equilibrium in the game induced by $\langle M, h\rangle$. Thus, for
 all $(s^1, \cdots,s^n)\in S_1\times \cdots\times S_n$, we have
 $$h(\sigma^1(s_1),\cdots,\sigma^n(s_n))\in F(s_1,\cdots,s_n)$$
 Furthermore, implementability in ex post strategies means that for all $i\in N$, $s\in S$, and $\rho^i:S_i\mapsto M_i$,
 \begin{align}\label{expost}
 v_i(h(\sigma^1(s_1),\cdots,\sigma^i(s_i),\cdots,\sigma^n(s_n)),s)\geqslant v_i(h(\sigma^1(s_1),\cdots,\rho^i(s_i),\cdots,\sigma^n(s_n)),s)
 \end{align}
Consider the following direct mechanism $(S_1\times\cdots\times S_n, g)$ where for all $(s_1,\cdots,s_n)\in S_1\times\cdots\times S_n$,
$$g(s_1,\cdots,s_n)=h(\sigma^1(s_1),\cdots,\sigma^n(s_n))\in F(s_1,\cdots,s_n)$$
It suffices to show that in the game induced by $(S_1\times\cdots\times S_n, g)$, it is ex post incentive compatible for each agent 
$i$ with type $s_i$ to report $s_i$. Suppose not. Then there is a profile $(s_1,\cdots,s_n)\in S_1\times\cdots\times S_n$ and an
agent $i\in N$ and a type $q\in S_i$ such that
$$v_i(g(q,s_{-i}),s)>v_i(g(s),s)$$
$$\Longleftrightarrow$$
$$v_i(h(\sigma^1(s_1),\cdots,\sigma^i(q),\cdots,\sigma^n(s_n)),s)>v_i(h(\sigma^1(s_1),\cdots,\sigma^i(s_i),\cdots,\sigma^n(s_n)),s)$$
Choose any $\rho^i:S_i\mapsto M_i$ such that $\rho^i(s_i)=\sigma^i(q)$. Then, the last inequality can be written as
$$v_i(h(\sigma^1(s_1),\cdots,\rho^i(s_i),\cdots,\sigma^n(s_n)),s)>v_i(h(\sigma^1(s_1),\cdots,\sigma^i(s_i),\cdots,\sigma^n(s_n)),s)$$
which contradicts inequality~\ref{expost}.
 \end{proof}