% Appendix A

\chapter{Revelation principles} % Main appendix title

\label{Appendix_A} % For referencing this appendix elsewhere, use \ref{AppendixA}

\begin{thm*}
revelation principle:

if a Mechanism $\langle M, h\rangle$ implements the social criteria 
$F$ in dominant strategy
equilibrium. Then there is a direct revelation mechanism which
implements $F$ truthfully in dominant strategy equilibrium(truth
telling is a dominant strategy equilibrium). 

\end{thm*}
\begin{proof}
 Since $F$ can be implemented in dominant strategies by  $\langle M, h\rangle$, there is a profile of strategies $(\sigma^1,\cdots,
 \sigma^n)\in (E_1\mapsto M_1)\times \cdots\times (E_n\mapsto M_n)$
 that forms an dominant strategy  equilibrium in the game induced by $\langle M, h\rangle$. Thus, for
 all $e=(e^1, \cdots,e^n)\in E=E_1\times \cdots\times E_n$, we have
 $$h(\sigma^1(e_1),\cdots,\sigma^n(e_n))\in F(e_1,\cdots,e_n)$$
 Furthermore, implementability in dominant strategies means that  in
 the mechanism $\langle M, h\rangle$, for
 all $i\in N$, $e\in E$, and any  strategy profile
 $\rho=(\rho^1,\cdots,\rho^n) in (E_1\mapsto M_1)\times \cdots\times (E_n\mapsto M_n)$,
 \begin{align}\label{domi}
 h(\rho^1(e_1),\cdots,\sigma^i(e_i),\cdots,\rho^n(e_n)) \succeq_{e_i} h(\rho^1(e_1),\cdots,\rho^i(e_i),\cdots,\rho^n(e_n))
 \end{align}
Consider the following direct mechanism $(E_1\times\cdots\times E_n, g)$ where for all $(e_1,\cdots,e_n)\in E_1\times\cdots\times E_n$,
$$g(e_1,\cdots,e_n)=h(\sigma^1(e_1),\cdots,\sigma^n(e_n))\in F(e_1,\cdots,e_n)$$
It suffices to show that in the game induced by
$(E_1\times\cdots\times E_n, g)$, it is a dominant strategy  for each agent 
$i$ with type $e_i$ to report $e_i$. Suppose not. Then there is a
profile $e=(e_1,\cdots, e'_i, \cdots, e_n)\in E_1\times\cdots \times
E_i \times \cdots \times E_n$ and an
agent $i\in N$ and another type $e'_i$ such that
$$g(e_1, \cdots, e'_i, \cdots, e_n) \succ_{e_i} g(e)$$
$$\Longleftrightarrow$$
$$h(\sigma^1(e_1),\cdots,\sigma^i(e'_i),\cdots,\sigma^n(e_n))\succ_{e_i} h(\sigma^1(e_1),\cdots,\sigma^i(e_i),\cdots,\sigma^n(e_n))$$
Choose  $\rho=(\rho^1,\cdots,\rho^i,\cdots,
 \rho^n)\in (E_1\mapsto M_1)\times \cdots\times (E_i\mapsto M_i)
 \times \cdots\times (E_n\mapsto M_n)$ such that
 $\rho^j(e_j)=\sigma^j(e_j)$ for every $j \in N, j\not = i$ and $\rho^i(e_i)=\sigma^i(e'_i)$. Then, the last inequality can be rewritten as
$$h(\rho^1(e_1),\cdots,\rho^i(e_i),\cdots,\rho^n(e_n))\succ_{e_i}h(\rho^1(e_1),\cdots,\sigma^i(e_i),\cdots,\rho^n(e_n))$$
which contradicts inequality \ref{domi}.
 \end{proof}

\begin{prop*}
revelation principle in an interdependent value environment:

if a Mechanism $\langle M, h\rangle$ implements the social choice rule $F$ in ex post 
equilibrium. Then there is a direct revelation mechanism which implements $F$ truthfully in ex post equilibrium(truth telling is a ex
post equilibrium). 

\end{prop*}
\begin{proof}
 Since $F$ can be implemented in ex post strategies by  $\langle M, h\rangle$, there is a profile of strategies $(\sigma^1,\cdots,
 \sigma^n)\in (S_1\mapsto M_1)\times \cdots\times (S_n\mapsto M_n)$ that forms an ex post equilibrium in the game induced by $\langle M, h\rangle$. Thus, for
 all $(s^1, \cdots,s^n)\in S_1\times \cdots\times S_n$, we have
 $$h(\sigma^1(s_1),\cdots,\sigma^n(s_n))\in F(s_1,\cdots,s_n)$$
 Furthermore, implementability in ex post strategies means that for all $i\in N$, $s\in S$, and $\rho^i:S_i\mapsto M_i$,
 \begin{align}\label{expost}
 v_i(h(\sigma^1(s_1),\cdots,\sigma^i(s_i),\cdots,\sigma^n(s_n)),s)\geqslant v_i(h(\sigma^1(s_1),\cdots,\rho^i(s_i),\cdots,\sigma^n(s_n)),s)
 \end{align}
Consider the following direct mechanism $(S_1\times\cdots\times S_n, g)$ where for all $(s_1,\cdots,s_n)\in S_1\times\cdots\times S_n$,
$$g(s_1,\cdots,s_n)=h(\sigma^1(s_1),\cdots,\sigma^n(s_n))\in F(s_1,\cdots,s_n)$$
It suffices to show that in the game induced by $(S_1\times\cdots\times S_n, g)$, it is ex post incentive compatible for each agent 
$i$ with type $s_i$ to report $s_i$. Suppose not. Then there is a profile $(s_1,\cdots,s_n)\in S_1\times\cdots\times S_n$ and an
agent $i\in N$ and a type $q\in S_i$ such that
$$v_i(g(q,s_{-i}),s)>v_i(g(s),s)$$
$$\Longleftrightarrow$$
$$v_i(h(\sigma^1(s_1),\cdots,\sigma^i(q),\cdots,\sigma^n(s_n)),s)>v_i(h(\sigma^1(s_1),\cdots,\sigma^i(s_i),\cdots,\sigma^n(s_n)),s)$$
Choose any $\rho^i:S_i\mapsto M_i$ such that $\rho^i(s_i)=\sigma^i(q)$. Then, the last inequality can be written as
$$v_i(h(\sigma^1(s_1),\cdots,\rho^i(s_i),\cdots,\sigma^n(s_n)),s)>v_i(h(\sigma^1(s_1),\cdots,\sigma^i(s_i),\cdots,\sigma^n(s_n)),s)$$
which contradicts inequality~\ref{expost}.
 \end{proof}