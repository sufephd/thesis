% Appendix B

\chapter{A Simple Proof to Hurwicz Impossibility Theorem} % Main appendix title

\label{Appendix_B} % For referencing this appendix elsewhere, use \ref{Appendix_B}

\begin{thm*}
(Hurwicz Impossibility Theorem) For the neoclassical pri-
vate goods economies, there is no mechanism < M, h > that implements Pareto efficient
and individually rational allocations in dominant strategy. Consequently, any revelation
mechanism < M, h > that yields Pareto efficient and individually rational allocations is
not strongly individually incentive compatible. (Truth-telling about their preferences is not
Nash Equilibrium).
\end{thm*}
The proof is adapted from the original proof.
\begin{proof}

  By the Revelation Principle, we only need to show that any
  revelation mechanism cannot implement Pareto efficient and
  individually rational allocations truthfully in dominant equilibrium
  for a particular pure exchange economy.

Then, it is enough
to show that truth-telling is not a Nash equilibrium for any revelation mechanism that
yields Pareto efficient and individually rational allocations for a particular pure exchange
economy.

Consider a private goods economy with two agents and  two goods. The
endowments are 

$$w_1=(2,0), w_2=(0,2).$$

The utilities are

$$ u_i(x,y) = \begin{cases}
3x_i + y_i & \text{if $x_i\leq y_i$ } \\
x_i + 3y_i & \text{if $x_i>y_i$ }
\end{cases}$$

The above things form the economic environment $e$.

For the pure exchange economy, it has the following
allocation results set 
$$ A =\{( (x_1,y_1), (x_2,y_2))| x_1+x_2 = 2 \ and\ y_1+y_2=2 \}$$

Let $U$ be the set of all neoclassical utility functions, i.e. they are continuous and quasi-
concave, which agent i can report to the designer. Thus, the true utility function
$u_i \in U$.

Then, 
$$ h: U\times U \rightarrow A $$

Suppose that  the true utility function profile
$u_i$ were indeed a Nash Equilibrium, it would then  satisfy

\[ u_i(h(u_i, u_{-i})) \geq u_i(h(u'_i,u_{-i}))\]

\begin{tikzpicture}
\draw (0,0) rectangle (8,8);
\draw (0,0) -- (8,8);
\draw (8,0) -- (2,2)--(0,8);
\draw (8,0) -- (6,6)--(0,8);
\draw (0,8)--(8,0);
\node [below] at (2,2) {$a$};
\node [right] at (6,6) {$b$};
\node [right] at (4, 4) {$f$};
\node [below left] at (0,0) {$O_1$};
\node [above right] at (8,8) {$O_2$};


\end{tikzpicture}

Let us denote the individual rational allocations by $IR(e)$, the
Pareto efficient allocations by $P(e)$, the true report allocation
$d=h(u_i, u_{-i})$.

First thing to note is that the Pareto efficient allocation results is
all 
on the 45 degree line. For any other point, move on to the 45 degree
line along the shortest path( the path that is orthogonal to it) is a
Pareto improvement. $P(e)=O_1O_2$.  
Individual rational means that the final allocation must be at least
as good as the endowment, so $P(e) \cap IR(e) = \overline{ab}$ .

Suppose $d \in P(e) \cap IR(e)$, that is, $d \in \overline{ab}$.

If agent 1 misreports his utility function :

$$ u_1(x,y)=x_1 + y_1$$

Then, with the misreported $u_1$ and thus the misreported $e'$, the new set of individually rational and Pareto efficient allocations is given
by $P(e') \cap IR(e') = \overline{fb}$


If agent 2 misreports his utility function :

$$ u_2(x,y)=x_2 + y_2$$

Then, with the misreported $u_2$ and thus the misreported $e''$, the new set of individually rational and Pareto efficient allocations is given
by $P(e'') \cap IR(e'') = \overline{af}$

 If the $d$ is on $\overline{af}$, agent 1 will choose to
misreport a $u_1(x,y)=x_1 + ky_1$ where $0<k<1$. For any $d$ that is on $\overline{fb}$, agent 2 will choose to
misreport a $u_2(x,y)=x_2 + ky_2$ where $0<k<1$. 

Thus, no mechanism that yields Pareto efficient and individually rational allocations is
incentive compatible for both sides.


\end{proof}



